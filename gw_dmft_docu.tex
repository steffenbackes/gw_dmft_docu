\documentclass[12pt,a4paper]{scrartcl}
\usepackage[utf8]{inputenc}
\usepackage{amsmath}
\usepackage{amssymb}
\usepackage{braket}
\usepackage{amssymb}
\usepackage{graphicx}
\usepackage{float}
\usepackage{wrapfig}
\usepackage{dsfont}
\usepackage{enumitem}
\usepackage{subfig}
\usepackage[top=3cm, left=3cm, right=3cm, bottom=3cm]{geometry}
\usepackage{fancyhdr}
\usepackage{sidecap}
\usepackage{pstricks}
\usepackage{mathrsfs}
\usepackage{listings}
\usepackage[absolute]{textpos}
\numberwithin{equation}{section}
%\numberwithin{figure}{section}
%\usepackage[pdfborder=0 0 0]{hyperref}
\usepackage[colorlinks=True, urlcolor=blue]{hyperref}
\pagestyle{fancy}
\fancyhead[C]{}
\fancyhead[L]{}
\fancyfoot[C]{\thepage}
\fancyhead[R]{}

\newcommand{\cng}[1]{{\color{red}#1}}
\newcommand{\sgn}{{\mathrm{sgn}}}
\newcommand{\GF}{Green's function}
\newcommand{\unity}{\mathds{1}}
\renewcommand{\vec}{\mathbf}

\title{GW (oneshot) + DMFT documentation}

\begin{document}
\maketitle
\tableofcontents


\section{Introduction}

This document  provides the prescription of the combination of a GW calculation
for correlated materials with DMFT applied only to a subset of correlated orbitals.
At this level, the GW calculation will be performed only once at the beginning (one-shot)
based on a DFT Hamiltonian $H^{DFT}$,
to obtain a nonlocal Selfenergy $\Sigma^{GW}$ for all states. 
In addition, the local part of $\Sigma^{GW}$ of a subset of strongly correlated orbitals will be replaced by a Selfenergy $\Sigma^{DMFT}$
obtained within a selfconsistent DMFT scheme, 
where the selfconsistency is done including the full nonlocal
effects of the combined Selfenergy.

No further selfconsistency apart from the DMFT cycle will be performed, {\it i.e.} no update of $\Sigma^{GW}$ will be done. By this, the final interacting system will be
described by a {\GF} with the non-interacting DFT dispersion, corrected by a non-local Selfenergy where the non-local components
correspond to $\Sigma^{GW}=G_0W_0$, while the local components
of the correlated orbitals correspond to $\Sigma^{DMFT}$.
This $\Sigma^{DMFT}$ is usually different to the one obtained by
a standard DMFT calculation since the selfconsistency is
done with the inclusion of the nonlocal parts of the 
Selfenergy.

Extensions to a full GW+DMFT selfconsistency will be discussed elsewhere.

\section{Approximation to the free energy functional $\Gamma[G,W]$: Combination of GW and DMFT}
As stated by Almbladh[ref], the free energy of a solid can be written in terms of
a functional $\Gamma[G,W]$ of the fully dressed Green's function $G$
and the screened Coulomb interaction $W$. While an analytic expression for 
$\Gamma$ is not known, it can be shown that it can be separated into 
a Hartree part $\Gamma^H$ and a correction arising from all other many-body effects
$\Psi$
\begin{align}
\Gamma[G,W] &= \Gamma^H[G,W] + \Psi[G,W].
\end{align}
The many-body correction $\Psi[G,W]$ is the sum of all skeleton diagrams that are
irreducible with respect to both one-electron propagator and
interaction lines. It has the properties
\begin{align}
\frac{\delta \Psi}{\delta G} &= \Sigma^{XC}  \label{eq:sigma_from_psi} \\
\frac{\delta \Psi}{\delta W} &= -\frac{1}{2}P \label{eq:p_from_psi},
\end{align}
where $\Sigma^{XC}$ is the exchange-correlation Selfenergy corresponding to the
fully dressed Green's function $G$, thus excluding the Hartree part $\Sigma^H$.
$P$ is the full polarization of the system that screens the bare Coulomb
interaction $V$ down to the screened interaction $W$.

Since methods like Density Functional Theory already treat the Hartree contribution
from the Coulomb interaction, we are interested in obtaining
an (approximative) expression for the many-body correction $\Psi[G,W]$.

One possibility is the GW approximation, which expands $\Psi[G,W]$ in powers of the
screened interaction $W$ and truncates the series at first order. The resulting expression
is thus
\begin{align}
\Psi[G,W] \approx -\frac{1}{2} \mathrm{Tr}(GWG).
\end{align}
Using equations \eqref{eq:sigma_from_psi} and \eqref{eq:p_from_psi},
we immediately obtain the $GW$ Selfenergy and polarization as
\begin{align}
\Sigma^{GW} &= -GW \\
P^{GW} &= GG.
\end{align}
While this approximation goes well beyond the level of a simple Hartree approximation
and usually treats all states without any separation of spaces,
it is only an expansion up to first order in $W$ and thus justified only when $W$ is
small, i.e. in case of weakly correlated systems. Thus, it is tempting to combine
$GW$ with other methods like DMFT for an improved treatment of correlated systems.

\bigskip

In the GW+DMFT scheme, we first separate the $\Psi$ functional into
its local and nonlocal parts
\begin{align}
\Psi[G,W] &= \Psi_{\mathrm{nonloc}}[G,W] + \Psi_{\mathrm{loc}}[G,W],
\end{align}
\cng{Is this exact?}
where usually the nonlocal part is approximated by $GW$, while the local part is usually approximated by DMFT,
but right now we do not want to decide on a specific method and only focus
on how to separate the local and nonlocal contributions.

We only want to impose one specific condition: First, we work in an orbital separated scheme,
where we separate the full Hilbert space $L+H$ into a correlated subspace $L$ and the remaining subspace $H$.
Then, our defintion of $\Psi_{\mathrm{loc}}$ is the following: $\Psi_{\mathrm{loc}}$ is generated only from the \underline{local}
components of $G$ and $W$ in the \underline{correlated subspace}, i.e.
\begin{align}
\Psi[G,W] &= \Psi_{\mathrm{nonloc}}[G,W] + \underbrace{\Psi_{}[G^{\mathrm{loc},L},W^{\mathrm{loc},L}]}_{\Psi_{\mathrm{loc}}}.
\label{eq:psi_separation_nonloc_loc}
\end{align}
By this, all internal processes contributing to $\Psi_{\mathrm{loc}}$ 
are restricted to the smaller correlated subspace $L$ and its local $G$ and $W$.
This construction already points to the usage of DMFT for $\Psi_{\mathrm{loc}}$, but it is instructive
not to fix on a specific method yet.
% 
% In general, in an orbitally separated scheme Eq. \eqref{eq:psi_separation_nonloc_loc} is not an 
% 
% 
% 
% 
% 
% \begin{align}
% \Psi[G,W] &\approx \Psi^{GW}_{\mathrm{nonloc}}[G,W] + \Psi^{DMFT}_{\mathrm{loc}}[G,W].
% \end{align}
All other contributions to the full $\Psi$ are now \underline{defined} to be originating from $\Psi_{\mathrm{nonloc}}[G,W]$.
We now have to explain what we actually mean by the two objects 
$\Psi_{\mathrm{nonloc}}$ and $\Psi_{\mathrm{loc}}$.

% \bigskip
% 
% Let us start first with the local part: In general, 
% $\Psi^{DMFT}_{\mathrm{loc}}$ is a functional of the fully dressed local Green's function,
% which in DMFT is obtained from an impurity model
% \begin{align}
%  \Psi^{DMFT}_{\mathrm{loc},L}[G^{\mathrm{imp},L},W^{\mathrm{imp},L}],
% \end{align}
% so in the evaluation of $\Psi^{DMFT}_{\mathrm{loc},L}$ all internal processes are generated
% from local objects. 

\bigskip

First, let us start with the nonlocal part $\Psi_{\mathrm{nonloc}}$.
Rewriting Eq. \eqref{eq:psi_separation_nonloc_loc} in the following way naturally
leads us to its definition via
\begin{align}
 \Psi[G,W] &= \Psi_{\mathrm{nonloc}}[G,W] + \Psi_{}[G^{\mathrm{loc},L},W^{\mathrm{loc},L}] \\
 &= \underbrace{\Psi[G,W]- \Psi[G^{\mathrm{loc},L},W^{\mathrm{loc},L}]}_{:=\Psi_{\mathrm{nonloc}}}
    + \Psi_{}[G^{\mathrm{loc},L},W^{\mathrm{loc},L}].
\end{align}
By this, we immediately see that applying any approximation $A$ to $\Psi_{\mathrm{nonloc}}$ and $\Psi_{\mathrm{loc}}$
will give us the approximate functional $\Psi^A = \Psi^A_{\mathrm{nonloc}}+\Psi^A_{\mathrm{loc}}$.

\bigskip

Side remark: Using the same approximation on both terms will not create any doublecounting or loss
of terms, regardless of whether we use an orbital separated scheme or not.

\bigskip

It starts to get really interesting when we use two different approximations $A$ and $B$
to treat the two terms
\begin{align}
\Psi[G,W] &\approx \Psi^A_{\mathrm{nonloc}}[G,W] + \Psi^B_{\mathrm{loc}}[G^{\mathrm{loc},L},W^{\mathrm{loc},L}] .
\end{align}
The main point here is \cng{IS THERE REALLY NO OVERLAP/DOUBLECOUNTING IN THIS SCHEME?}

\bigskip

In the context of GW+DMFT we will now approximate the nonlocal part by GW and the local part
by DMFT. The GW approximation is usually performed as a single-shot, and thus
based on the DFT Green's function $G^0$ and RPA screened interaction $W^0$,
while the local functional from DMFT is obtained from the impurity Green's function and
interaction, i.e.
\begin{align}
\Psi[G,W] &\approx \Psi^{GW}_{\mathrm{nonloc}}[G,W] + \Psi^{DMFT}_{\mathrm{loc}}[G^{\mathrm{loc},L},W^{\mathrm{loc},L}] \\
&= -\frac{1}{2}\Big( G^0W^0G^0 - G^{0,\mathrm{loc},L}W^{0,\mathrm{loc},L}G^{0,\mathrm{loc},L}  \Big)
                    + \Psi^{DMFT}[G^{\mathrm{imp},L},W^{\mathrm{imp},L}] .
\end{align}
Please note that in the orbital separated scheme 
\begin{align}
 \Sigma^{GW,\mathrm{loc},L} \neq \sum_k \frac{\delta}{\delta G^L_k}
             \Big( -\frac{1}{2} G^{0,\mathrm{loc},L}W^{0,\mathrm{loc},L}G^{0,\mathrm{loc},L} \Big),
\end{align}
but this is not relevant here.
Using equations \eqref{eq:sigma_from_psi} and \eqref{eq:p_from_psi} we obtain for the GW+DMFT
Selfenergy and polarization
\begin{align}
\Sigma_{ab} &= 
\begin{cases}
-[G^0W^0]_{ab} + [G^{0,\mathrm{loc},L}W^{0,\mathrm{loc},L}]_{ab}+ \Sigma_{ab}^{\mathrm{imp},XC,L} & \mbox{ for } a,b \in L  \\
-[G^0W^0]_{ab}                                                                      & \mbox{ for } a \mbox{ or/and } b \in H 
\end{cases}\\
%
%
P_{abcd} &= 
\begin{cases}
-[G^0G^0]_{abcd} + [G^{0,\mathrm{loc},L}G^{0,\mathrm{loc},L}]_{abcd}+ P_{abcd}^{\mathrm{imp},L} & \mbox{ for } a,b,c,d \in L  \\
-[G^0G^0]_{abcd}                                                                      & \mbox{ for } \{a,b,c,d\} \cap H  \neq \emptyset
\end{cases}
\end{align}

Limiting cases:
\begin{description}
\item[$W$ small:] In this case the Selfenergy will be well described already by GW, so the impurity solution
will basically give the same result $G^{0,\mathrm{loc},L}W^{0,\mathrm{loc},L} = \Sigma^{\mathrm{imp},L}$.
The two terms cancel and we fully regain the GW result. \cng{NO doublecounting like in FLL LDA+U or LDA+DMFT!}
\item[W large:] Then the GW Selfenergy contribution within the subspace $L$ is basically given by the impurity solution.
There is no mismatch of exchange terms originating from outside the space $L$, since only the local impurity exchange
is removed from GW and replaced by the local impurity DMFT exchange. But this term can be different due to rearrangements
of the local impurity charge. This contribution is not yet considered in the GW screening since no GW selfconsistency
has been applied.
\cng{Can we now do another GW calculation only replacing the new contribution from the impurity states?
Then we perform a GW selfconsistency including the  local vertex corrections from DMFT, in similar spirit as Boehnke et al., but additionally with improved transitions between
$L$ and $H$ which they do not update.}
\end{description}


%%%%%%%%%%%%%%%%%%%%%%%%%%%%%%%%%%%%%%%%%%%%%%%%%%%%%%%%%%%%%%%%%%%%%%%%%%%%%%%%%%%%%%%%%%%%%%%%
%%%%%%%%%%%%%%%%%%%%%%%%%%%%%%%%%%%%%%%%%%%%%%%%%%%%%%%%%%%%%%%%%%%%%%%%%%%%%%%%%%%%%%%%%%%%%%%%
%%%%%%%%%%%%%%%%%%%%%%%%%%%%%%%%%%%%%%%%%%%%%%%%%%%%%%%%%%%%%%%%%%%%%%%%%%%%%%%%%%%%%%%%%%%%%%%%

\section{Product basis}
In the GW formalism we encounter objects such as the inverse dielectric function
\begin{align}
 \epsilon^{-1},
\end{align}
which is a two-particle operator. The standard way to specify the action of a two-particle operator
is to start from a complete orthonormal single-particle basis $\{ \ket{i} \}$, where
\begin{align}
 \braket{\vec{r} | i} &= \psi_i(\vec{r}),
\end{align}
with a complex valued function $\psi_i: R^3\rightarrow \mathbb{C}$.
Then one introduces a two-particle basis $\{ \ket{ij} \}$, which is composed of the single-particle states via
\begin{align}
\braket{\vec{r}\vec{r}' | ij} 
&= \Big( \bra{\vec{r}} \otimes \bra{\vec{r}'} \Big) \Big( \ket{i} \otimes \ket{j} \Big) \\
&= \psi_i(\vec{r})\psi_j(\vec{r}').
\end{align}
In this basis, any two-particle operator $A$ can be represented as a rank-4 tensor
by its action on the two-particle basis states
\begin{align}
A_{ijkl} &= \braket{ij | A | kl }  \\
&= \int \int \mathrm{d}\vec{r}\mathrm{d}\vec{r}' \, \psi^*_i(\vec{r})\psi^*_j(\vec{r}') 
                                          A(\vec{r},\vec{r}') \psi_l(\vec{r}')\psi_k(\vec{r}),
\end{align}
where we have assumed that 
\begin{align}
 \braket{\vec{r}\vec{r}' | A | \vec{r}''\vec{r}''' }
 &= A(\vec{r},\vec{r}') \delta(\vec{r}-\vec{r}'') \delta(\vec{r}'-\vec{r}'''),
\end{align}
which applies to the Coulomb interaction operator and all other operators we will consider here.

Our goal is to obtain a matrix (rank-2) representation of the two-particle operator $A$, so that we can 
define a proper inverse $A^{-1}$ or a multiplication $AB$ between these operators. 
This is usually done in two ways:

%%%%%%%%%%%%%%%%%%%%%%%%%%%%%%%%%%%%%%%%%%%%%%%%%%%%%%%%%%%%%%%%%%%%%%%%%%%%%
\subsection{Index combination}
In the two-particle basis the structure of the Tensor-elements
\begin{align}
 A_{ijkl} &= \braket{ij | A | kl } ,
\end{align}
suggests that we could also interpret each basis state as
\begin{align}
 \ket{a} := \ket{ ij } &= \ket{i}\times \ket{j} ,
\end{align}
with a single basis state $\ket{a}$, where the index $a$ now runs over $N^2$ values
if we have $N$ single particle states $\ket{i}$. In this notation, we can indeed write the tensor elements
as matrix elements
\begin{align}
 A_{ab} &= \braket{a | A | b } \\
 &= \braket{ij | A | kl }   \\
  &= A_{(ij)(kl)} .
\end{align}

\subsubsection{Properties and consistency}

The combination of the two left, the ``outgoing'' indices $ij$ and the right, the ``incoming'' $kl$ is in principle arbitrary.
We could also combine the indices $ik$ of the first partcile and $jl$ of the second particle. 
Though, the $(ij)(kl)$ combination should be preferrable, since it does not mix vectors $\ket{i}$ with their
dual counterpart $\bra{i}$. 

Furthermore, in cases where we want to apply this scheme, the tensor operations can indeed be rewritten
as a matrix multiplication in the combined ``ingoing-outgoing'' index notation. For example
the screened interaction $W$ is given by
\begin{align}
W_{ijkl} &= [ v + vPW ]_{ijkl} \\
&= v_{ijkl} + \sum_{mnop} v_{ijmn}P_{mnop}W_{opkl},
\end{align}
with the bare interaction $v$ and the polarization $P$. Using the combined index notation we get the representation
\begin{align}
W_{ab} &= W_{(ij)(kl)} \\
&= v_{(ij)(kl)} + \sum_{(mn)(op)} v_{(ij)(mn)}P_{(mn)(op)}W_{(op)(kl)} \\
&= v_{ab} + \sum_{cd} v_{ac}P_{cd}W_{da} \\
&= [v + vPW]_{ab},
\end{align}
where $vPW$ is to be understood as the matrix product of $v,P$ and $W$ in the combined index notation.

\subsubsection{Tensor inverse}

Now let us try to obtain a closed expression of $W$ satisfying this expression, which
is not possible in the 4-index tensor notation.
For this, we need to define first an object $\unity$ that serves as the identity element, which then will
allow us to poperly define an inverse of a matrix in the combined index notation.
The identity element should have the following property
\begin{align}
 \unity A &= A =  A \unity ,
\end{align}
where $A$ is a two-particle tensor, which means
\begin{align}
A_{(ij)(kl)} &= A_{ab} \\
&= [\unity A]_{ab} \\
&= \sum_c \unity_{ac} A_{ca} \\
&= \sum_{mn} \unity_{(ij)(mn)} A_{(mn)(kl)} .
\end{align}
From this we conclude that 
\begin{align}
 \unity_{(ij)(mn)} &= \delta_{im}\delta_{jn} \\
 \Rightarrow \unity_{ac} &= \delta_{ac},
\end{align}
which leads to the natural definition of the identity element in the combined index notation.
It can be directly seen that also $ A \unity = A$ is fulfilled.

From this we can define the inverse $A^{-1}$ as the standard matrix inverse in the combined index notation
that fulfills
\begin{align}
A^{-1}A &= AA^{-1} = \unity,
\end{align}
since
\begin{align}
[A^{-1}A]_{(ij)(kl)} &= [A^{-1}A]_{ab} \\
&= \delta_{ab} \\
&= \delta_{ik}\delta_{kl} \\
&= \unity_{(ij)(kl)}
\end{align}
With this, we can finally solve the equation above for the screened interaction
\begin{align}
 W &= v + vPW \\
 \Rightarrow (\unity - vP) W &= v \\
 \Rightarrow  W &=(\unity - vP)^{-1} v ,
\end{align}
By contruction, the tensor elements $W_{ijkl} =[(\unity - vP)^{-1} v]_{ijkl}$
will now satisfy the equation above for the screened interaction.

\subsubsection{Problems}
Possible problems are:
\begin{itemize}
\item A two-particle operator diagonal in position representation cannot be inverted
in the combined index-notation! This can be for example a purely local Coulomb interaction!

Consider 
\begin{align}
 \braket{\vec{r}\vec{r}' | A | \vec{r}''\vec{r}''' }
 &= A(\vec{r}) \delta(\vec{r}-\vec{r}') \delta(\vec{r}-\vec{r}'') \delta(\vec{r}'-\vec{r}'''),
\end{align}
and we assume that $A(\vec{r})=a>0$ constant, \textit{i.e.} it does not matter where the particles interact.
As long as their positions are identical they pick up a factor $a$.

This leads to the following tensor elements
\begin{align}
A_{ijkl} &= \braket{ij | A | kl }  \\
&= \int \int \mathrm{d}\vec{r}\mathrm{d}\vec{r}' \, \psi^*_i(\vec{r})\psi^*_j(\vec{r}') 
                                          a \delta(\vec{r}-\vec{r}') \psi_l(\vec{r}')\psi_k(\vec{r}) \\
&=a  \int \mathrm{d}\vec{r} \, \psi^*_i(\vec{r})\psi^*_j(\vec{r}) 
                                          \psi_l(\vec{r})\psi_k(\vec{r}) .
\end{align}
For the case of real wave functions we see that we always get a non-zero contribution when we pair 
two indices with one another, leading to an integral of the form
\begin{align}
a  \int \mathrm{d}\vec{r} \, \psi^2_m(\vec{r}) \psi^2_n(\vec{r}) \neq 0
\end{align}
For the example of $N=2$ single-particle states $\psi_i(\vec{r})$, we can choose two out of four indices to pair, then
the last one has to be paired with $N=2$ choices for the index, leading to 8 combinations which are at least non-zero.
In the combined index notation we then can arrive at the following matrix 
(\cng{Note: I have confirmed this numerically for a few basis sets})
\begin{align}
 A &= a
 \begin{pmatrix}
  c_1 & 0 & 0 & c_3 \\
  0   & c_2 & c_2 & 0 \\
  0 & c_2 & c_2 & 0 \\
  c_3 & 0 & 0 & c_4 
 \end{pmatrix},
\end{align}
where we can immediately see that this matrix cannot be inverted since two columns are linearly dependent (even identical).

\item A ``constant'' operator $A(\vec{r},\vec{r}')=c\neq 0$  can be inverted! (See discussion in next section)

\end{itemize}


%%%%%%%%%%%%%%%%%%%%%%%%%%%%%%%%%%%%%%%%%%%%%%%%%%%%%%%%%%%%%%%%%%%%%%%%%%%%%

\subsection{Aryasetiawan-style}
\subsubsection{Defining the new basis set}
We start by rewriting the equation for the tensor elements in the two-particle basis in the 
following way 
\begin{align}
 \braket{ij | A | kl }  
&= \int \int \mathrm{d}\vec{r}\mathrm{d}\vec{r}' \, \psi^*_i(\vec{r})\psi^*_j(\vec{r}') 
                                          A(\vec{r},\vec{r}') \psi_l(\vec{r}')\psi_k(\vec{r}) \\
%
&= \int \int \mathrm{d}\vec{r}\mathrm{d}\vec{r}' \, \psi^*_i(\vec{r})\psi_k(\vec{r}) 
                                          A(\vec{r},\vec{r}') \psi^*_j(\vec{r}')\psi_l(\vec{r}') \\
%
&= \int \int \mathrm{d}\vec{r}\mathrm{d}\vec{r}' \, \underbrace{ \Big( \psi^*_k(\vec{r}) \psi_i(\vec{r}) \Big)^* }_{B^*_a(\vec{r})}
                                          A(\vec{r},\vec{r}') \underbrace{ \psi^*_j(\vec{r}')\psi_l(\vec{r}') }_{B_b(\vec{r} )} \\
&= \braket{ B_a | A | B_b } .
\end{align}
From these observation we see that we should define new basis states $\{ B_a \}$ by the product of the single-particle
states by
\begin{align}
 \braket{ \vec{r} | B_a } := \psi^*_i(\vec{r}) \psi_j(\vec{r}),
\end{align}
where the index $a:=a(i,j)$ lables the combination of the indices $i,j$. As a result, for a finite number of single-particle
states $N=\mathrm{dim}\{ \ket{i} \}$, we have $N^2$ new basis states $\{ B_a \}$, and could define the relation between the indices
as
\begin{align}
 a(i,j) := iN + j.
\end{align}
Though, the new set $\{ B_a \}$ is \textit{not} a basis, since it is overcomplete/linearly dependent and in addition not orthonormal.
For example if there exist at least two real single-particle basis functions $\psi_i,\psi_j$, one has 
\begin{align}
\psi_i(\vec{r})\psi_j(\vec{r}) = B_a(\vec{r}) = B_b(\vec{r}) = \psi_j(\vec{r})\psi_i(\vec{r}),
\end{align}
where $a\neq b$.
Also, for a states with indices $a(i,i)$ and $b(i,i)$, which leads to
\begin{align}
 \braket{B_a | B_b} &= \int \mathrm{d}\vec{r} \, \psi^*_i(\vec{r}) \psi_i(\vec{r}) \psi^*_j(\vec{r}) \psi_j(\vec{r}) \\
 &= \int \mathrm{d}\vec{r} \, |\psi_i(\vec{r})|^2 |\psi_j(\vec{r})|^2 
\end{align}
which is usually larger than zero for $a\neq b$, i.e. $i=j$ and not equal to $1$ in case $a=b$, i.e. $i\neq j$.

The overlap matrix for the new set $\{ B_a \}$ is therefore different from the unit matrix.
If we define 
\begin{align}
 v := 
 \begin{pmatrix}
%   \ket{B_1}\\ \ket{B_2}\\\ket{B_3}\\ \vdots \\ \ket{B_{N^2}}
\ket{B_1} & \ket{B_2} & \ket{B_3}& \cdots & \ket{B_{N^2}}  
 \end{pmatrix},
\end{align}
then the overlap matrix can be written as
\begin{align}
O &= v^{\dagger}\cdot v,
\end{align}
since
\begin{align}
O_{ab} &= (v^{\dagger}\cdot v)_{ab} \\
 &= \braket{B_a | B_b} .
\end{align}
Due to the overcompleteness, some columns of $O$ will not be linear independent, i.e. some of the Eigenvalues 
$\lambda_i$ of $O$ will be equal to zero.

\subsubsection{Reduction of the basis set and reorthonormalization}

After diagonalizing $O$ to obtain the Eigenvectors $o_1,o_2,...,o_{N^2}$, we throw away the ones with Eigenvalue zero 
and set up the matrices
\begin{align}
 U &=
 \begin{pmatrix}
  o_1 & o_2 & o_3 & \cdots o_{N_r}
 \end{pmatrix} \in \mathbb{C}^{N^2 \times N_r},
\end{align}
where $N_r \leq N^2$ is the number of Eigevectors with nonzero Eigenvalue,
and
\begin{align}
 D^{-1/2} &:= \mathrm{diag}( {\lambda_1}^{-1/2}, \ {\lambda_2}^{-1/2}, \ {\lambda_3}^{-1/2}, \ \cdots,\ {\lambda_{N_r}}^{-1/2}  ) 
 \in \mathbb{R}^{N_r \times N_r},
\end{align}
where $\lambda_i$ are the nonzero Eigenvalues of $O$.


The elements of the following vector
\begin{align}
\tilde{v} := vUD^{-1/2} ,
\end{align}
will then yield a new set of $N_r$ basis functions which are complete and orthonormal, as can be seen from
the new overlap matrix
\begin{align}
\tilde{O} &= \tilde{v}^{\dagger}\cdot \tilde{v} \\
&= D^{-1/2}U^{\dagger} (v^{\dagger} \cdot v) UD^{-1/2} \\
&= D^{-1/2} \underbrace{U^{\dagger} O U}_{=D} D^{-1/2} \\
&= \unity  \in \mathbb{R}^{N_r \times N_r}.
\end{align}

\textit{Remark:}
To reduce the size of the basis for computational efficiency, one can also exclude Eigenvectors with a finite, 
but small Eigenvalue, i.e. below some treshhold $\delta > \lambda_i$.
Then the new basis will only be approximately complete and orthonormal, but the error can be controlled by choosing
$\delta$ sufficiently small.

\subsubsection{Product basis matrix elements}

After having obtained the new complete orthonormal basis $\{ \tilde{B}_a  \}$, the matrix elements
of any two-particle operator are then given as
\begin{align}
A_{ab}
&= \braket{ \tilde{B}_a | A | \tilde{B}_b } \\
&= \int \int \mathrm{d}\vec{r}\mathrm{d}\vec{r}' \,  \tilde{B}^*_a(\vec{r})  A(\vec{r},\vec{r}') \tilde{B}_b(\vec{r}').
\end{align}
The representation of $A$ in the product basis is then given as
\begin{align}
A &= \sum_{a,b} \ket{\tilde{B}_a}A_{ab}\bra{\tilde{B}_b} .
\end{align}
or in the position representation \cng{DOES THIS MAKE SENSE?}
\begin{align}
 A(\vec{r},\vec{r}') &= \sum_{a,b} \braket{\vec{r}|\tilde{B}_a} A_{ab}\braket{\tilde{B}_b|\vec{r}'} \\
 &= \sum_{a,b} \tilde{B}_a(\vec{r}) A_{ab}\tilde{B}^*_b(\vec{r}') .
\end{align}
\cng{CHECK HERE IF BOTH REPRESENTATIONS LEAD TO THE SAME RESULT!!! The product basis should recover
the right A(r-r'), shouldn't it? (I think so...) Can we actually prove this?}


\subsubsection{Switching between the product and the two-particle basis}
If we want to obtain the original tensor representation of a two-particle operator in the 
product basis, we have to evaluate \cng{DOES THIS MAKE SENSE?}
\begin{align}
A_{ijkl} &= \braket{ij | A | kl }  \\
&= \int \int \mathrm{d}\vec{r}\mathrm{d}\vec{r}' \, \psi^*_i(\vec{r})\psi^*_j(\vec{r}') 
                                          A(\vec{r},\vec{r}') \psi_l(\vec{r}')\psi_k(\vec{r}) \\
&= \sum_{a,b}\int \int \mathrm{d}\vec{r}\mathrm{d}\vec{r}' \, \psi^*_i(\vec{r})\psi^*_j(\vec{r}') 
         \tilde{B}_a(\vec{r}) A_{ab}\tilde{B}^*_b(\vec{r}') \psi_l(\vec{r}')\psi_k(\vec{r}).
\end{align}
The other way round, if we have a two-particle operator given in the two-particle basis, the product basis
representation can be obtained by
\begin{align}
A_{ab}
&= \braket{ \tilde{B}_a | A | \tilde{B}_b } \\
&= \int \int \mathrm{d}\vec{r}\mathrm{d}\vec{r}' \,  \tilde{B}^*_a(\vec{r})  A(\vec{r},\vec{r}') \tilde{B}_b(\vec{r}') \\
&=\sum_{ijkl} \int \int \mathrm{d}\vec{r}\mathrm{d}\vec{r}' \,  \tilde{B}^*_a(\vec{r})  
    \psi_i(\vec{r})\psi_j(\vec{r}') A_{ijkl} \psi^*_l(\vec{r}')\psi^*_k(\vec{r}) \tilde{B}_b(\vec{r}') .
\end{align}


% \begin{align}
%  \braket{ii | A | jj }  
% &= \int \int \mathrm{d}\vec{r}\mathrm{d}\vec{r}' \, \psi^*_i(\vec{r})\psi_j(\vec{r}) 
%                                           A(\vec{r},\vec{r}') \psi^*_i(\vec{r}')\psi_j(\vec{r}')
% \end{align}
% 
% \begin{align}
%  \braket{ij | A | ij }  
% &= \int \int \mathrm{d}\vec{r}\mathrm{d}\vec{r}' \, \psi^*_i(\vec{r})\psi_i(\vec{r}) 
%                                           A(\vec{r},\vec{r}') \psi^*_j(\vec{r}')\psi_j(\vec{r}')
% \end{align}
% 
% \begin{align}
%  \braket{ij | A | ji }  
% &= \int \int \mathrm{d}\vec{r}\mathrm{d}\vec{r}' \, \psi^*_i(\vec{r})\psi_j(\vec{r}) 
%                                           A(\vec{r},\vec{r}') \psi^*_j(\vec{r}')\psi_i(\vec{r}')
% \end{align}

\subsubsection{Tensor inverse}

Since in the product basis any two-particle operator can now be
represented as a matrix/rank-2 tensor, we can finally define its inverse $\tilde{A}^{-1}$ as the standard Matrix inverse, which then in the product basis fulfils the property
\begin{align}
 (\tilde{A}^{-1}A)_{ab}
&= \braket{ \tilde{B}_a | \tilde{A}^{-1}A | \tilde{B}_b } \\
&= \delta_{ab}.
\end{align}
In the position representation
\begin{align}
\braket{\vec{r} | \tilde{A}^{-1}A | \vec{r}' }
 &= \sum_{a,b} \braket{\vec{r}|\tilde{B}_a} (\tilde{A}^{-1}A)_{ab}\braket{\tilde{B}_b|\vec{r}'} \\
%
&= \sum_{a,b} \braket{\vec{r}|\tilde{B}_a} \delta_{ab} \braket{\tilde{B}_b|\vec{r}'} \\
&= \sum_{a} \braket{\vec{r}|\tilde{B}_a} \braket{\tilde{B}_a|\vec{r}'} \\
&= \braket{\vec{r}|\vec{r}'} \\
&= \delta(\vec{r}-\vec{r}'),
\end{align}
if the product basis is complete.

%\braket{\vec{r}|\tilde{B}_a} A_{a\braket{\tilde{B}_b|\vec{r}'}

\subsubsection{Problems}
Possible problems are:
\begin{itemize}
 \item A ``constant'' tensor of the form $A(\vec{r},\vec{r}')=c\neq 0$ cannot be inverted! 
 \cng{I'm not sure, this is maybe correct behaviour?}

 Let us assume we obtain a final basis state $\ket{\tilde{B}_o}$ in a one-dimensional system
 which is a real odd function in position representation, \textit{i.e.}
 \begin{align}
  \tilde{B}_o(r) &= -\tilde{B}_o(-r).
 \end{align}
For a constant tensor this leads to the following
 \begin{align}
A_{ab}
&=c \int \int \mathrm{d}\vec{r}\mathrm{d}\vec{r}' \,  \tilde{B}^*_a(\vec{r})  \tilde{B}_b(\vec{r}') \\
&=c \left(\int \mathrm{d}\vec{r} \, \tilde{B}^*_a(\vec{r}) \right) \left( \int \mathrm{d}\vec{r}'\,  \tilde{B}_b(\vec{r}'), \right),   
\end{align}
\textit{i.e.} the two integrations over the basis states decouple, and everytime $a$ or $b$ is equal to the
real odd function $\tilde{B}_o$, we obtain zero. Therefore, the $o$-th column and row is equal to zero.
Which means, we have at least one Eigenvalue equal to zero and, therefore, the tensor
in the product basis cannot be inverted.

\item The ``identity'' tensor of the form $\braket{ij|A|kl}=\delta_{ik}\delta_{jl}$ 
cannot be inverted! 
This can be seen from
\begin{align}
A_{ab}
&= \braket{ \tilde{B}_a | A | \tilde{B}_b } \\
&=\sum_{ijkl} \int \int \mathrm{d}\vec{r}\mathrm{d}\vec{r}' \,  \tilde{B}^*_a(\vec{r})  
    \psi_i(\vec{r})\psi_j(\vec{r}') A_{ijkl} \psi^*_l(\vec{r}')\psi^*_k(\vec{r}) \tilde{B}_b(\vec{r}') \\
 &=\sum_{ij} \int \int \mathrm{d}\vec{r}\mathrm{d}\vec{r}' \,  \tilde{B}^*_a(\vec{r})  
 \psi_i(\vec{r})\psi_j(\vec{r}') \psi^*_j(\vec{r}')\psi^*_i(\vec{r}) \tilde{B}_b(\vec{r}') \\
  &=\sum_{ij} \left(\int \mathrm{d}\vec{r} \,  \tilde{B}^*_a(\vec{r})  
 |\psi_i(\vec{r})|^2 \right)
 \left(\int \mathrm{d}\vec{r}' \,  \tilde{B}_b(\vec{r}')  
 |\psi_j(\vec{r}')|^2 \right).
\end{align}
 If one basis function $\tilde{B}_a(\vec{r})$ is an odd function, 
 the full column $a$ and row $a$ will be zero, so 
 we have at least one Eigenvalue equal to zero and, therefore, the tensor
in the product basis cannot be inverted.
\end{itemize}



%%%%%%%%%%%%%%%%%%%%%%%%%%%%%%%%%%%%%%%%%%%%%%%%%%%%%%%%%%%%%%%%%%%%%%%%%%%%%%%%%%%%%%%%%%%%%%%%
%%%%%%%%%%%%%%%%%%%%%%%%%%%%%%%%%%%%%%%%%%%%%%%%%%%%%%%%%%%%%%%%%%%%%%%%%%%%%%%%%%%%%%%%%%%%%%%%
%%%%%%%%%%%%%%%%%%%%%%%%%%%%%%%%%%%%%%%%%%%%%%%%%%%%%%%%%%%%%%%%%%%%%%%%%%%%%%%%%%%%%%%%%%%%%%%%


\section{The GW part}

On the basis of $H^{DFT}$ a $G_0W_0$ calculation has to be performed on the full
system. By this, the Selfenergy in the Kohn-Sham basis is obtained for all states
\begin{align}
 \Sigma_{\nu\nu'}(k,\omega)
 &= \left[ G_0W_0  \right]_{\nu\nu'}(k,\omega).
\end{align}
By this, the GW estimate for the full interacting {\GF} is given by
\begin{align}
 G^{GW}_{\nu\nu'}(k,\omega) 
 &= \left[ \unity(\omega +\mu +i\delta) -H^{DFT}(k)+v^{XC}(k) - \Sigma^{GW}(k,\omega)  \right]^{-1}_{\nu\nu'}.
\end{align}


\subsection{Output for DMFT}
At this point the basis transformation to the local Wannier basis will be
performed on the GW side.
For the next step of the DMFT calculation one needs
on a mesh in k-space in the full Brillouin zone:
\begin{itemize}
\item $\epsilon_m(k)$: The eigenvalues  of $H^{DFT}(k)$ in the Wannier basis
for all relevant orbitals
\item $\mu$: The chemical potential that yields the correct physical number of electrons $N_e$. It is not needed if all $\epsilon_m(k)$ are given 
with respect to the Fermi level.
for $H^{DFT}$
\item $v^{XC}_{mm'}(k)$: The value of the exchange-correlation potential 
in the Wannier basis for all relevant orbitals
\item $\Sigma^{GW}_{mm'}(k,i\omega)$: The Selfenergy within GW in the Wannier basis for all relevant orbitals on imaginary frequencies $\omega$.
\item $V_{abcd}(q)$: The bare Coulomb interaction elements in the Wannier basis for all relevant orbitals.
\item $P^{GW}_{abcd}(q,i\nu)$: The polarization  in the Wannier basis for all relevant orbitals on imaginary frequencies.
\item $\Sigma^{GW,imp}_{mm'}(i\omega) = -[G^{0,loc,L}W^{0,loc,L}]_{mm'}$
The selfenergy of the impurity model solved in GW, \textit{i.e.}
all indices and internal transitions restricted to the correlated subspace
\item $P^{GW,imp}_{abcd}(i\nu) = [G^{0,loc,L}G^{0,loc,L}]_{abcd}$
The polarization of the impurity model solved in GW, \textit{i.e.}
all indices and internal transitions restricted to the correlated subspace
\item $\beta$: The inverse temperature used for defining $\omega_n=(2n+1)\pi/\beta$.
\end{itemize}
{\it All output from the GW calculation will be in atomic units and
have to be converted to eV!!}


\section{The DMFT part}

Within DMFT we then calculate a local correction $\Sigma^{DMFT}$ for a subset of correlated
Wannier orbitals.

The input of the calculation will be the output of the GW calculation. First, one will usually apply a Wannier-interpolation of the GW data to obtain a fine k-mesh
since the GW output will be given on a very coarse grid.

Since the Hartree term is already incorporated on the DMFT level, 
it has to be excluded from the Selfenergies in GW (already taken care of)
and in DMFT (has to be done in the selfconsistency).

\subsection{The self-consistency cycle}

We then proceed as follows:

\begin{enumerate}

%\item Transform the Selfenergy from GW to the imaginary Matsubara axis by
%\begin{align}
%\Sigma^{GW}_{mm'}(k,i\omega_n)
%&= \frac{1}{2\pi i} \int \, \frac{\Sigma^{GW}_{mm'}(k,\omega)}{\omega - i\omega_n}\ \mathrm{d}\omega , \mbox{\hspace{1cm} or}\\
%&= \frac{1}{\pi} \int \, \frac{\mathrm{Im}[\Sigma^{GW}_{mm'}(k,\omega)]}{\omega - i\omega_n} \ \mathrm{d}\omega , \mbox{\hspace{0.7cm} or} \\
%&= \frac{1}{\pi i} \int \, \frac{\mathrm{Re}[\Sigma^{GW}_{mm'}(k,\omega)]}{\omega - i\omega_n} \ \mathrm{d}\omega .
%\end{align}

%\item Calculate the local diagonal part of the GW Selfenergy ONLY for the correlated subset of orbitals $f$ that will be later replaced by the DMFT result
%\begin{align}
% \Sigma^{GW,loc}_{f}(i\omega_n)
% &= \frac{1}{N_k}\sum_k \Sigma^{GW}_{ff}(k,i\omega_n).
%\end{align}


%\item (This step can be omitted if the correct physical number of electrons $N_e$ is already known)
% 
%Construct the initial non-interacting {\GF} on the imaginary Matsubara axis via
%\begin{align}
%   G_{0,mm'}(k,i\omega_n) 
% &= \left[ \unity(i\omega_n +\mu) -H^{DFT}(k)   \right]^{-1}_{mm''},
%\end{align}
% and calculate the total number of electrons
%\begin{align}
%  N_e
% &= \lim_{\tau\rightarrow 0^-} \frac{1}{\beta N_k} 
%                \sum_{i\omega_n}\sum_{k,m}G_{0,mm}(k,i\omega_n) \mathrm{e}^{-i\omega_n\tau}.
%\end{align}

\item Make a first guess for the local DMFT impurity Selfenergy  $\Sigma^{imp}$ and polarization $P^{imp}$, for example one can use the GW result
\begin{align}
\Sigma^{imp}(i\omega_n) &=  \Sigma^{GW,imp}(i\omega_n) \\
P^{imp}(i\nu_n) &=  P^{GW,imp}(i\nu_n) .
\end{align}
Please note that $\Sigma$ is a matrix in the orbital basis and $P$ is a tensor. 
We will work in the combined index notation to treat $P$ and all further tensors as standard matrices.

\item Set up the interacting {\GF} $G$ and the screened interaction $W$, where the impurity component of the GW contribution for the correlated orbitals is replaced by the DMFT contribution.
%In the first step when using $\Sigma^{imp}_{ff} =  \Sigma^{GW,loc}_{f}$ this 
%term is zero.
\begin{align}
 G(k,i\omega_n) 
 &= \left[ \unity(i\omega_n+\mu ) -H^{DFT}(k) + v^{XC}(k) \right.\\
          & \hspace{1cm}- \Sigma^{GW}(k,i\omega_n) 
          + \Sigma^{GW,imp}(i\omega_n)
          \left. - \Sigma^{imp}(i\omega_n)
           \right]^{-1} \\
%
W(q,i\nu_n) &= \left[ V^{-1}(q) - P^{GW}(q,i\nu_n) + P^{GW,imp}(i\nu_n) 
                              -P^{imp}(i\nu_n)\right]^{-1}
\end{align}
Adjust the chemical potential $\mu$ in a way such that the desired
filling
\begin{align}
  N_e
 &= \lim_{\tau\rightarrow 0^-} \frac{1}{\beta N_k} 
                \sum_{i\omega_n}\sum_{k,m}G_{mm}(k,i\omega_n) \mathrm{e}^{-i\omega_n\tau}.
\end{align}
is obtained.

\item Calculate the local {\GF}  and the local screened interaction (for all orbitals) 
\begin{align}
 G^{loc}(i\omega_n) &= \frac{1}{N_k}\sum_k  G(k,i\omega_n) \\
 W^{loc}(i\nu_n) &= \frac{1}{N_k}\sum_q  W(q,i\nu_n).
\end{align}
and then derive the Weiss field $\mathscr{G}$ and the effective interaction ONLY
on the subset of correlated orbitals in the impurity model. Neglecting
offdiagonal components in the hybridization we are also only interested 
in the orbital diagonal components.
% using the local GW Selfenergy where the diagonal components of the
% correlated orbitals $f$ are replaced by the impurity Selfenergy
\begin{align}
\mathscr{G}(i\omega_n) &= \left[ [G^{loc} ]^{-1}(i\omega_n) + \Sigma^{imp}(i\omega_n)\right]^{-1} \\
\mathcal{U}(i\nu_n) &= \left[ [W^{loc} ]^{-1}(i\nu_n) + P^{imp}(i\nu_n)\right]^{-1}.
\end{align}

The Weiss field matrix $\mathscr{G}$ is not explicitly needed, so it is not necessary to
invert the equation for $\mathscr{G}$ .

\item Calculate the Hybridization function
\begin{align}
 \Delta(i\omega_n)
 &= i\omega_n + \tilde{\mu} -  \mathscr{G}^{-1}(i\omega_n),
\end{align}
where the local chemical potential $\tilde{\mu}$ (which is orbital dependent!) is given by 
\begin{align}
 \tilde{\mu} = \lim_{\omega_n\rightarrow \infty} \mathrm{Re}\left[ \mathscr{G}^{-1}(i\omega_n) \right]
\end{align}
and transform $\Delta(i\omega_n)$ to the imaginary time axis $\tau$ by a Fourier transform
\begin{align}
 \Delta(\tau) &= \frac{1}{\beta} \sum_{i\omega_n} \Delta(i\omega_n) \mathrm{e}^{-i\omega_n\tau}.
\end{align}
Generate the $K(\tau),K'(\tau)$ functions from the retarded interaction via ($\mathcal{U}(i\nu_n)$
is real!)
\begin{align}
K(\tau) &= -\frac{2}{\beta} \sum_{n>0} \frac{\mathcal{U}(i\nu_n)-\mathcal{U}(0)}{\nu^2_n}
                                        ( \cos(\nu_n\tau)-1 ) \\
K'(\tau) &= \frac{2}{\beta} \sum_{n>0} \frac{\mathcal{U}(i\nu_n)-\mathcal{U}(0)}{\nu_n}
                                        \sin(\nu_n\tau),
\end{align}
and solve the impurity model for the correlated $f$ orbitals.

\item Obtain the new local Selfenergy $\Sigma^{imp}(i\omega_n)$ and the
susceptibility $\chi^{imp}_{abcd}(i\nu_n)$ from the impurity model.
Calculate the updated impurity polarization via 
\begin{align}
P^{imp}(i\nu_n) &= -[\unity - \chi^{imp}(i\nu_n)\mathcal{U}(i\nu_n)]^{-1}\chi^{imp}(i\nu_n),
\end{align}
% and calculate the updated 
%Hartree correction from the impurity occupations as given by Eq. \eqref{eq:hartree_selfenergy}
%and \eqref{eq:hartree_selfenergy_freq}.
%full {\GF}
%\begin{align}
% G_{mm'}(k,i\omega_n) 
% &= \left[ \unity(i\omega_n+\mu ) -H^{DFT}(k) + v^{XC}_{mm'}(k) \right.\\
%          &- \Sigma^{GW}(k,i\omega_n) 
%          + \Sigma^{GW,loc}_f(i\omega_n)
%          \left. - \Sigma^{imp}(i\omega_n) \right]^{-1}_{mm'}.
%\end{align}
%\begin{align}
%\Sigma^{H,imp}_{f\sigma}
%&=  U_{ff}(n_{f\sigma}+n_{f\bar{\sigma}})
%+ \sum_{f'\neq f} U'_{ff'}n_{f'\bar{\sigma}}
%+ \sum_{f'\neq f} (U'_{ff'}-J_{ff'}) n_{f'\sigma}.
%\end{align}
%Adjust the chemical potential $\mu$ in such a way that the number of electrons
%$N_e$ is preserved.
Then go back to step 2. Repeat until convergence
is reached.

\end{enumerate}


\subsection{Output}
After convergence, {\it e.g.} the local spectral function $A_m(\omega)$
can be obtained by analytic continuation of $ G^{loc}_{mm}(i\omega_n) $.


\section{Hartree- and Exchange term in DMFT}
The derivation follows the ideas of Haule PRL 115, 196403 (2015).

\subsection{Hartree term}
The Hartree energy has the general form
\begin{align}
E^H[\rho]
%
&= \frac{1}{2}\int \mathrm{d}\vec{r}\mathrm{d}\vec{r}' \, 
\rho(\vec{r}) V_C(\vec{r}-\vec{r}') \rho(\vec{r}') \\
%
&= \frac{1}{2} \int \mathrm{d}\vec{r}\mathrm{d}\vec{r}' \,
\frac{\rho(\vec{r}) \rho(\vec{r}') }{|\vec{r}-\vec{r}'|},
\end{align}
where $\rho(\vec{r})$ is the sum of all spin-components
\begin{align}
\rho(\vec{r}) &= \rho_{\uparrow}(\vec{r}) + \rho_{\downarrow}(\vec{r}).
\end{align}

In order to evaluate these term for DMFT we introduce a local orbital basis  $\ket{\chi^{\sigma}_m}$, 
and replace the bare Coulomb interaction $V_C(\vec{r}-\vec{r}')$ 
by an effective screened Coulomb interaction $V_{DMFT}(\vec{r}-\vec{r}')$.
This leads to 
\begin{align}
E^H[\rho]
%
&= \frac{1}{2}\int \mathrm{d}\vec{r}\mathrm{d}\vec{r}' \, 
\rho(\vec{r}) V_C(\vec{r}-\vec{r}') \rho(\vec{r}') \\
%
&= \frac{1}{2}\sum_{\substack{klmn\\\sigma\sigma'}}\int \mathrm{d}\vec{r}\mathrm{d}\vec{r}' 
\braket{\vec{r}|\chi^{\sigma}_k}  \braket{\chi^{\sigma}_k|\rho|\chi^{\sigma}_l}  
\braket{\chi^{\sigma}_l | \vec{r}}V_{DMFT}(\vec{r}-\vec{r}') \nonumber \\
& \hspace{2.7cm} \times 
\braket{\vec{r}'|\chi^{\sigma'}_m}  \braket{\chi^{\sigma'}_m|\rho|\chi^{\sigma'}_n}  
\braket{\chi^{\sigma'}_n | \vec{r}'} \\
%
&= \frac{1}{2}\sum_{\substack{klmn\\\sigma\sigma'}}\int \mathrm{d}\vec{r}\mathrm{d}\vec{r}' 
 (\chi^{\sigma}_l)^*(\vec{r}) (\chi^{\sigma'}_n)^*(\vec{r}')
    V_{DMFT}(\vec{r}-\vec{r}') 
  \chi^{\sigma'}_m(\vec{r}')  \chi^{\sigma}_k(\vec{r}) \nonumber \\
& \hspace{2.7cm} \times 
\braket{\chi^{\sigma}_k|\rho|\chi^{\sigma}_l} 
\braket{\chi^{\sigma'}_m|\rho|\chi^{\sigma'}_n}  .
\end{align}
%
In the last equation we can now identify the matrix elements of the local screened Coulomb
interaction
\begin{align}
\braket{ln|U|km}
&=
\int \mathrm{d}\vec{r}\mathrm{d}\vec{r}' 
 (\chi^{\sigma}_l)^*(\vec{r}) (\chi^{\sigma'}_n)^*(\vec{r}')
    V_{DMFT}(\vec{r}-\vec{r}') 
  \chi^{\sigma'}_m(\vec{r}')  \chi^{\sigma}_k(\vec{r}) ,
\end{align}
and the DMFT density matrix
\begin{align}
\braket{\chi^{\sigma}_k|\rho|\chi^{\sigma}_l} 
&= n^{\sigma}_{kl}.
\end{align}
With this, the Hartree energy takes on the form
\begin{align}
E^{DMFT}
&= \frac{1}{2}\sum_{\substack{klmn\\\sigma\sigma'}}
\braket{ln|U|km}
n^{\sigma}_{kl} n^{\sigma'}_{mn}.
\end{align}
In the impurity model we restrict ourselves to diagonal density matrices, which leads to
\begin{align}
E^H_{DMFT}
&= \frac{1}{2}\sum_{\substack{km\\\sigma\sigma'}}
\braket{km|U|km}
n^{\sigma}_{k} n^{\sigma'}_{m}.
\end{align}
This leads to the following Hartree part of the Selfenergy in DMFT
\begin{align}
\Sigma^{H,DMFT}_{l\sigma}
&= \frac{\partial }{\partial n^{\sigma}_l} E^H_{DMFT} \\
&= \frac{1}{2} \sum_{m,\sigma'} \braket{lm|U|lm} n^{\sigma'}_{m}
+
\frac{1}{2} \sum_{k,\sigma'} \braket{kl|U|kl} n^{\sigma'}_{k} \\
%
&= \sum_{m,\sigma'} \braket{lm|U|lm} n^{\sigma'}_{m} \\
%
&= U_0 (n^{\uparrow}_l + n^{\downarrow}_l)
               + \sum_{m\neq l} (U_0-2J_{lm}) (n^{\uparrow}_m + n^{\downarrow}_m) .
\label{eq:hartree_selfenergy}
\end{align}


\subsection{Exchange term}
The exact exchange energy has the general form
\begin{align}
E^X[\rho]
%
&= -\frac{1}{2}\sum_{\sigma} \int \mathrm{d}\vec{r}\mathrm{d}\vec{r}' \, 
\rho_{\sigma}(\vec{r},\vec{r}') V_C(\vec{r}-\vec{r}') \rho_{\sigma}(\vec{r}',\vec{r}) \\
%
&= -\frac{1}{2}\sum_{\sigma} \int \mathrm{d}\vec{r}\mathrm{d}\vec{r}' \,
\frac{\rho_{\sigma}(\vec{r},\vec{r}') \rho_{\sigma}(\vec{r}',\vec{r}) }{|\vec{r}-\vec{r}'|}.
\end{align}
In order to evaluate these term for DMFT we introduce a local orbital basis  $\ket{\chi^{\sigma}_m}$, 
and replace the bare Coulomb interaction $V_C(\vec{r}-\vec{r}')$ 
by an effective screened Coulomb interaction $V_{DMFT}(\vec{r}-\vec{r}')$.
This leads to 
\begin{align}
E^X[\rho]
&= -\frac{1}{2}\sum_{\sigma} \int \mathrm{d}\vec{r}\mathrm{d}\vec{r}' \, 
\rho_{\sigma}(\vec{r},\vec{r}') V_{DMFT}(\vec{r}-\vec{r}') \rho_{\sigma}(\vec{r}',\vec{r}) \\
%
&= -\frac{1}{2} \sum_{\substack{klmn\\\sigma}} \int \mathrm{d}\vec{r}\mathrm{d}\vec{r}' \, 
\braket{\vec{r} | \chi^{\sigma}_k }\braket{\chi^{\sigma}_k | \rho_{\sigma} | \chi^{\sigma}_l } \braket{\chi^{\sigma}_l | \vec{r}'}  V_{DMFT}(\vec{r}-\vec{r}') \nonumber\\
& \hspace{3.8cm} \times 
\braket{\vec{r}' | \chi^{\sigma}_m }\braket{\chi^{\sigma}_m | \rho_{\sigma} | \chi^{\sigma}_n } \braket{\chi^{\sigma}_n | \vec{r}} \\
% 
&= -\frac{1}{2} \sum_{\substack{klmn\\\sigma}} \int \mathrm{d}\vec{r}\mathrm{d}\vec{r}' \, 
(\chi^{\sigma}_n)^*(\vec{r}) (\chi^{\sigma}_l)^*(\vec{r}')  
V_{DMFT}(\vec{r}-\vec{r}') 
\chi^{\sigma}_m (\vec{r}') \chi^{\sigma}_k (\vec{r}) \nonumber\\
& \hspace{3.8cm} \times 
\braket{\chi^{\sigma}_m | \rho_{\sigma} | \chi^{\sigma}_n }
\braket{\chi^{\sigma}_k | \rho_{\sigma} | \chi^{\sigma}_l }.
\end{align}
In the last equation we can now identify the matrix elements of the local screened Coulomb
interaction
\begin{align}
\braket{nl|U|km} 
&= \int \mathrm{d}\vec{r}\mathrm{d}\vec{r}' \, 
(\chi^{\sigma}_n)^*(\vec{r}) (\chi^{\sigma}_l)^*(\vec{r}')  
V_{DMFT}(\vec{r}-\vec{r}') 
\chi^{\sigma}_m (\vec{r}') \chi^{\sigma}_k (\vec{r}),
\end{align}
and the DMFT density matrix
\begin{align}
\braket{\chi^{\sigma}_m | \rho_{\sigma} | \chi^{\sigma}_n } 
&= n^{\sigma}_{mn}.
\end{align}
With this, the exchange energy takes on the form
\begin{align}
E^X_{DMFT}
&= -\frac{1}{2} \sum_{\substack{klmn\\\sigma}}\braket{nl|U|km} n^{\sigma}_{mn} n^{\sigma}_{kl}.
\end{align}
In the impurity model we restrict ourselves to diagonal density matrices, which leads to
\begin{align}
E^X_{DMFT}
&= -\frac{1}{2} \sum_{mk,\sigma}\braket{mk|U|km} n^{\sigma}_{m} n^{\sigma}_{k} .
\end{align}
This leads to the following exchange part of the Selfenergy in DMFT
\begin{align}
\Sigma^{X,DMFT}_{l\sigma}
&= \frac{\partial }{\partial n^{\sigma}_l} E^X_{DMFT} \\
%
&= -\frac{1}{2} \sum_{k}\braket{lk|U|kl} n^{\sigma}_{k} 
   -\frac{1}{2} \sum_{m}\braket{ml|U|lm} n^{\sigma}_{m}  \\
%
&= - \sum_{k}\braket{lk|U|kl} n^{\sigma}_{k} \\
%
&= - U_0\, n^{\sigma}_l -  \sum_{k\neq l} J_{lk} \, n^{\sigma}_{k}.
\end{align}

\subsection{Hartree + exchange Selfenergy}
For consistency checks, we add up the Hartree and the exchange contribution
to the Selfenergy and obtain
\begin{align}
\Sigma^{H,DMFT}_{l\sigma} + \Sigma^{X,DMFT}_{l\sigma}
%
&= U_0 n^{\bar{\sigma}}_l 
%
               + \sum_{m\neq l} (U_0-2J_{lm}) n^{\bar{\sigma}}_m  \nonumber \\
& \hspace{1.4cm} + \sum_{m\neq l} (U_0-3J_{lm}) n^{     \sigma }_m  \\
&= \lim_{\omega_n\rightarrow \infty} \Sigma^{DMFT}(i\omega_n),
\end{align}
which is identical to the high-frequency limit of the true DMFT Selfenergy.
This term is also equal to the sum of all first order diagrams to the
DMFT Selfenergy, i.e. the Hartree- and the Fock diagram.

\subsection{Dynamical interactions}
In the case of dynamical interactions $U(\omega)$ in the Hartree and exchange part
the screened Coulomb matrix elements recover their bare values \cng{is this correct?}, i.e.
$U_0$ has to be replaced by the bare $V$ (assuming no frequency dependence of the 
Hund's coupling)
\begin{align}
\Sigma^{H,DMFT}_{l\sigma}
&= V (n^{\uparrow}_l + n^{\downarrow}_l)
               + \sum_{m\neq l} (V-2J_{lm}) (n^{\uparrow}_m + n^{\downarrow}_m) 
\label{eq:hartree_selfenergy_freq} \\
%
\Sigma^{X,DMFT}_{l\sigma}
&= - V\, n^{\sigma}_l -  \sum_{k\neq l} J_{lk} \, n^{\sigma}_{k}.               
\end{align}
\cng{CAUTION!} Does $U_0$ or $F_0$ recover the bare interaction? If $F_0=V$,
then $U_0$ has to be replaced in the 5-orbital model by $V + \frac{8}{7}J_{avg}$.


\section{Implementation details}

\subsection{Impurity solver input}

The CT-HYB impurity solver by Yusuke needs the following input files
\begin{description}
\item[dmft.input] Includes information about U,J, number of frequencies, etc. At the moment
possible: Only 3-fold degenerate orbitals. No freq. dependent U.

\item[hyb\_tau.dat] The hybridization function as a matrix for imaginary time. 
It needs to be diagonal! \\
Only real part, one column. Seperate matrix elements
via two line breaks and \verb|# hyb     2    1| etc. We need \verb|Nmesh+1|
points where the endpoints $\tau=0,\beta$ are included! By convention has negative sign.
\cng{The local orbital levels are assumed to be $\tilde{\mu}=0$ and any
shift is absorbed in the chemical potential! This has to 
be checked for consistency!!!}

\item[omega\_mesh.dat] Specifies the bosonic frequency grid for some 
correlation functions. Just reuse the standard template file. Not important for us.

\item[fort.10*] Includes information about the Monte-Carlo configuration used
for starting the sampling. Is initialized once with Yusuke's code and then overwritten by 
the solver. No change required here.
\end{description}


\end{document}
