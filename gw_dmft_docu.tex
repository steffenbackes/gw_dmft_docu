\documentclass[12pt,a4paper]{scrartcl}
\usepackage[utf8]{inputenc}
\usepackage{amsmath}
\usepackage{amssymb}
\usepackage{braket}
\usepackage{amssymb}
\usepackage{graphicx}
\usepackage{float}
\usepackage{wrapfig}
\usepackage{dsfont}
\usepackage{enumitem}
\usepackage{subfig}
\usepackage[top=3cm, left=3cm, right=3cm, bottom=3cm]{geometry}
\usepackage{fancyhdr}
\usepackage{sidecap}
\usepackage{pstricks}
\usepackage{mathrsfs}
\usepackage{listings}
\usepackage[absolute]{textpos}
\numberwithin{equation}{section}
%\numberwithin{figure}{section}
%\usepackage[pdfborder=0 0 0]{hyperref}
\usepackage[colorlinks=True, urlcolor=blue]{hyperref}
\pagestyle{fancy}
\fancyhead[C]{}
\fancyhead[L]{}
\fancyfoot[C]{\thepage}
\fancyhead[R]{}

\newcommand{\cng}[1]{{\color{red}#1}}
\newcommand{\sgn}{{\mathrm{sgn}}}
\newcommand{\GF}{Green's function}
\newcommand{\unity}{\mathds{1}}
\renewcommand{\vec}{\mathbf}

\title{GW (oneshot) + DMFT documentation}

\begin{document}
\maketitle
\tableofcontents


\section{Introduction}

This document  provides the prescription of the combination of a GW calculation
for correlated materials with DMFT applied only to a subset of correlated orbitals.
At this level, the GW calculation will be performed only once at the beginning (one-shot)
based on a DFT Hamiltonian $H^{DFT}$,
to obtain a nonlocal Selfenergy $\Sigma^{GW}$ for all states. 
In addition, the local part of $\Sigma^{GW}$ of a subset of strongly correlated orbitals will be replaced by a Selfenergy $\Sigma^{DMFT}$
obtained within a selfconsistent DMFT scheme, 
where the selfconsistency is done including the full nonlocal
effects of the combined Selfenergy.

No further selfconsistency apart from the DMFT cycle will be performed, {\it i.e.} no update of $\Sigma^{GW}$ will be done. By this, the final interacting system will be
described by a {\GF} with the non-interacting DFT dispersion, corrected by a non-local Selfenergy where the non-local components
correspond to $\Sigma^{GW}=G_0W_0$, while the local components
of the correlated orbitals correspond to $\Sigma^{DMFT}$.
This $\Sigma^{DMFT}$ is usually different to the one obtained by
a standard DMFT calculation since the selfconsistency is
done with the inclusion of the nonlocal parts of the 
Selfenergy.

Extensions to a full GW+DMFT selfconsistency will be discussed elsewhere.

\section{Approximation to the free energy functional $\Gamma[G,W]$: Combination of GW and DMFT}
As stated by Almbladh[ref], the free energy of a solid can be written in terms of
a functional $\Gamma[G,W]$ of the fully dressed Green's function $G$
and the screened Coulomb interaction $W$. While an analytic expression for 
$\Gamma$ is not known, it can be shown that it can be separated into 
a Hartree part $\Gamma^H$ and a correction arising from all other many-body effects
$\Psi$
\begin{align}
\Gamma[G,W] &= \Gamma^H[G,W] + \Psi[G,W].
\end{align}
The many-body correction $\Psi[G,W]$ is the sum of all skeleton diagrams that are
irreducible with respect to both one-electron propagator and
interaction lines. It has the properties
\begin{align}
\frac{\delta \Psi}{\delta G} &= \Sigma^{XC}  \label{eq:sigma_from_psi} \\
\frac{\delta \Psi}{\delta W} &= -\frac{1}{2}P \label{eq:p_from_psi},
\end{align}
where $\Sigma^{XC}$ is the exchange-correlation Selfenergy corresponding to the
fully dressed Green's function $G$, thus excluding the Hartree part $\Sigma^H$.
$P$ is the full polarization of the system that screens the bare Coulomb
interaction $V$ down to the screened interaction $W$.

Since methods like Density Functional Theory already treat the Hartree contribution
from the Coulomb interaction, we are interested in obtaining
an (approximative) expression for the many-body correction $\Psi[G,W]$.

One possibility is the GW approximation, which expands $\Psi[G,W]$ in powers of the
screened interaction $W$ and truncates the series at first order. The resulting expression
is thus
\begin{align}
\Psi[G,W] \approx -\frac{1}{2} \mathrm{Tr}(GWG).
\end{align}
Using equations \eqref{eq:sigma_from_psi} and \eqref{eq:p_from_psi},
we immediately obtain the $GW$ Selfenergy and polarization as
\begin{align}
\Sigma^{GW} &= -GW \\
P^{GW} &= GG.
\end{align}
While this approximation goes well beyond the level of a simple Hartree approximation
and usually treats all states without any separation of spaces,
it is only an expansion up to first order in $W$ and thus justified only when $W$ is
small, i.e. in case of weakly correlated systems. Thus, it is tempting to combine
$GW$ with other methods like DMFT for an improved treatment of correlated systems.

\bigskip

In the GW+DMFT scheme, we first separate the $\Psi$ functional into
its local and nonlocal parts
\begin{align}
\Psi[G,W] &= \Psi_{\mathrm{nonloc}}[G,W] + \Psi_{\mathrm{loc}}[G,W],
\end{align}
\cng{Is this exact?}
where usually the nonlocal part is approximated by $GW$, while the local part is usually approximated by DMFT,
but right now we do not want to decide on a specific method and only focus
on how to separate the local and nonlocal contributions.

We only want to impose one specific condition: First, we work in an orbital separated scheme,
where we separate the full Hilbert space $L+H$ into a correlated subspace $L$ and the remaining subspace $H$.
Then, our defintion of $\Psi_{\mathrm{loc}}$ is the following: $\Psi_{\mathrm{loc}}$ is generated only from the \underline{local}
components of $G$ and $W$ in the \underline{correlated subspace}, i.e.
\begin{align}
\Psi[G,W] &= \Psi_{\mathrm{nonloc}}[G,W] + \underbrace{\Psi_{}[G^{\mathrm{loc},L},W^{\mathrm{loc},L}]}_{\Psi_{\mathrm{loc}}}.
\label{eq:psi_separation_nonloc_loc}
\end{align}
By this, all internal processes contributing to $\Psi_{\mathrm{loc}}$ 
are restricted to the smaller correlated subspace $L$ and its local $G$ and $W$.
This construction already points to the usage of DMFT for $\Psi_{\mathrm{loc}}$, but it is instructive
not to fix on a specific method yet.
% 
% In general, in an orbitally separated scheme Eq. \eqref{eq:psi_separation_nonloc_loc} is not an 
% 
% 
% 
% 
% 
% \begin{align}
% \Psi[G,W] &\approx \Psi^{GW}_{\mathrm{nonloc}}[G,W] + \Psi^{DMFT}_{\mathrm{loc}}[G,W].
% \end{align}
All other contributions to the full $\Psi$ are now \underline{defined} to be originiating from $\Psi_{\mathrm{nonloc}}[G,W]$.
We now have to define what we actually mean by the two objects 
$\Psi_{\mathrm{nonloc}}$ and $\Psi_{\mathrm{loc}}$.

% \bigskip
% 
% Let us start first with the local part: In general, 
% $\Psi^{DMFT}_{\mathrm{loc}}$ is a functional of the fully dressed local Green's function,
% which in DMFT is obtained from an impurity model
% \begin{align}
%  \Psi^{DMFT}_{\mathrm{loc},L}[G^{\mathrm{imp},L},W^{\mathrm{imp},L}],
% \end{align}
% so in the evaluation of $\Psi^{DMFT}_{\mathrm{loc},L}$ all internal processes are generated
% from local objects. 

\bigskip

First, let us start with the nonlocal part $\Psi_{\mathrm{nonloc}}$.
Rewriting Eq. \eqref{eq:psi_separation_nonloc_loc} in the following way naturally
leads us to its definition via
\begin{align}
 \Psi[G,W] &= \Psi_{\mathrm{nonloc}}[G,W] + \Psi_{}[G^{\mathrm{loc},L},W^{\mathrm{loc},L}] \\
 &= \underbrace{\Psi[G,W]- \Psi[G^{\mathrm{loc},L},W^{\mathrm{loc},L}]}_{:=\Psi_{\mathrm{nonloc}}}
    + \Psi_{}[G^{\mathrm{loc},L},W^{\mathrm{loc},L}].
\end{align}
By this, we immediately see that applying any approximation $A$ to $\Psi_{\mathrm{nonloc}}$ and $\Psi_{\mathrm{loc}}$
will give us the approximate functional $\Psi^A = \Psi^A_{\mathrm{nonloc}}+\Psi^A_{\mathrm{loc}}$.

\bigskip

Side remark: Using the same approximation on both terms will not create any doublecounting or loss
of terms, regardless of whether we use an orbital separated scheme or not.

\bigskip

It starts to get really interesting when we use two different approximations $A$ and $B$
to treat the two terms
\begin{align}
\Psi[G,W] &\approx \Psi^A_{\mathrm{nonloc}}[G,W] + \Psi^B_{\mathrm{loc}}[G^{\mathrm{loc},L},W^{\mathrm{loc},L}] .
\end{align}
The main point here is \cng{IS THERE REALLY NO OVERLAP/DOUBLECOUNTING IN THIS SCHEME??}

\bigskip

In the context of GW+DMFT we will now approximate the nonlocal part by GW and the local part
by DMFT. The GW approximation is usually performed as a single-shot, and thus
based on the DFT Green's function $G^0$ and RPA screened interaction $W^0$,
while the local functional from DMFT is obtained from the impurity Green's function and
interaction, i.e.
\begin{align}
\Psi[G,W] &\approx \Psi^{GW}_{\mathrm{nonloc}}[G,W] + \Psi^{DMFT}_{\mathrm{loc}}[G^{\mathrm{loc},L},W^{\mathrm{loc},L}] \\
&= -\frac{1}{2}\Big( G^0W^0G^0 - G^{0,\mathrm{loc},L}W^{0,\mathrm{loc},L}G^{0,\mathrm{loc},L}  \Big)
                    + \Psi^{DMFT}[G^{\mathrm{imp},L},W^{\mathrm{imp},L}] .
\end{align}
Please note that in the orbital separated scheme 
\begin{align}
 \Sigma^{GW,\mathrm{loc},L} \neq \sum_k \frac{\delta}{\delta G^L_k}
             \Big( -\frac{1}{2} G^{0,\mathrm{loc},L}W^{0,\mathrm{loc},L}G^{0,\mathrm{loc},L} \Big),
\end{align}
but this is not relevant here.
Using equations \eqref{eq:sigma_from_psi} and \eqref{eq:p_from_psi} we obtain for the GW+DMFT
Selfenergy and polarization
\begin{align}
\Sigma_{ab} &= 
\begin{cases}
-[G^0W^0]_{ab} + [G^{0,\mathrm{loc},L}W^{0,\mathrm{loc},L}]_{ab}+ \Sigma_{ab}^{\mathrm{imp},XC,L} & \mbox{ for } a,b \in L  \\
-[G^0W^0]_{ab}                                                                      & \mbox{ for } a \mbox{ or/and } b \in H 
\end{cases}\\
%
%
P_{abcd} &= 
\begin{cases}
-[G^0G^0]_{abcd} + [G^{0,\mathrm{loc},L}G^{0,\mathrm{loc},L}]_{abcd}+ P_{abcd}^{\mathrm{imp},L} & \mbox{ for } a,b,c,d \in L  \\
-[G^0G^0]_{abcd}                                                                      & \mbox{ for } \{a,b,c,d\} \cap H  \neq \emptyset
\end{cases}
\end{align}

Limiting cases:
\begin{description}
\item[$W$ small:] In this case the Selfenergy will be well described already by GW, so the impurity solution
will basically give the same result $G^{0,\mathrm{loc},L}W^{0,\mathrm{loc},L} = \Sigma^{\mathrm{imp},L}$.
The two terms cancel and we fully regain the GW result. \cng{NO doublecounting like in FLL LDA+U or LDA+DMFT!}
\item[W large:] Then the GW Selfenergy contribution within the subspace $L$ is basically given by the impurity solution.
There is no mismatch of exchange terms originating from outside the space $L$, since only the local impurity exchange
is removed from GW and replaced by the local impurity DMFT exchange. But this term can be different due to rearrangements
of the local impurity charge. This contribution is not yet considered in the GW screening since no GW selfconsistency
has been applied.
\cng{Can we now do another GW calculation only replacing the new contribution from the impurity states?
Then we perform a GW selfconsistency including the  local vertex corrections from DMFT, in similar spirit as Boehnke et al.}
\end{description}










\section{The GW part}

On the basis of $H^{DFT}$ a $G_0W_0$ calculation has to be performed on the full
system. By this, the Selfenergy in the Kohn-Sham basis is obtained for all states
\begin{align}
 \Sigma_{\nu\nu'}(k,\omega)
 &= \left[ G_0W_0  \right]_{\nu\nu'}(k,\omega).
\end{align}
By this, the GW estimate for the full interacting {\GF} is given by
\begin{align}
 G^{GW}_{\nu\nu'}(k,\omega) 
 &= \left[ \unity(\omega +\mu +i\delta) -H^{DFT}(k)+v^{XC}(k) - \Sigma^{GW}(k,\omega)  \right]^{-1}_{\nu\nu'}.
\end{align}


\subsection{Output for DMFT}
At this point the basis transformation to the local Wannier basis will be
performed on the GW side.
For the next step of the DMFT calculation one needs
on a mesh in k-space in the full Brillouin zone:
\begin{itemize}
\item $\epsilon_m(k)$: The eigenvalues  of $H^{DFT}(k)$ in the Wannier basis
for all relevant orbitals
\item $\mu$: The chemical potential that yields the correct physical number of electrons $N_e$. It is not needed if all $\epsilon_m(k)$ are given 
with respect to the Fermi level.
for $H^{DFT}$
\item $v^{XC}_{mm'}(k)$: The value of the exchange-correlation potential 
in the Wannier basis for all relevant orbitals
\item $\Sigma^{GW}_{mm'}(k,i\omega)$: The Selfenergy within GW in the Wannier basis for all relevant orbitals on imaginary frequencies $\omega$.
\item $\beta$: The inverse temperature used for defining $\omega_n=(2n+1)\pi/\beta$.
\end{itemize}
{\it All output from the GW calculation will be in atomic units and
have to be converted to eV!!}


\section{The DMFT part}

Within DMFT we then calculate a local correction $\Sigma^{DMFT}$ for a subset of correlated
Wannier orbitals.

The input of the calculation will be the output of the GW calculation. First, one will usually apply a Wannier-interpolation of the GW data to obtain a fine k-mesh
since the GW output will be given on a very coarse grid.

Since the Hartree term is already incorporated on the DMFT level, 
it has to be excluded from the Selfenergies in GW (already taken care of)
and in DMFT (has to be done in the selfconsistency).

\subsection{The self-consistency cycle}

We then proceed as follows:

\begin{enumerate}

%\item Transform the Selfenergy from GW to the imaginary Matsubara axis by
%\begin{align}
%\Sigma^{GW}_{mm'}(k,i\omega_n)
%&= \frac{1}{2\pi i} \int \, \frac{\Sigma^{GW}_{mm'}(k,\omega)}{\omega - i\omega_n}\ \mathrm{d}\omega , \mbox{\hspace{1cm} or}\\
%&= \frac{1}{\pi} \int \, \frac{\mathrm{Im}[\Sigma^{GW}_{mm'}(k,\omega)]}{\omega - i\omega_n} \ \mathrm{d}\omega , \mbox{\hspace{0.7cm} or} \\
%&= \frac{1}{\pi i} \int \, \frac{\mathrm{Re}[\Sigma^{GW}_{mm'}(k,\omega)]}{\omega - i\omega_n} \ \mathrm{d}\omega .
%\end{align}

\item Calculate the local diagonal part of the GW Selfenergy ONLY for the correlated subset of orbitals $f$ that will be later replaced by the DMFT result
\begin{align}
 \Sigma^{GW,loc}_{f}(i\omega_n)
 &= \frac{1}{N_k}\sum_k \Sigma^{GW}_{ff}(k,i\omega_n).
\end{align}


%\item (This step can be omitted if the correct physical number of electrons $N_e$ is already known)
% 
%Construct the initial non-interacting {\GF} on the imaginary Matsubara axis via
%\begin{align}
%   G_{0,mm'}(k,i\omega_n) 
% &= \left[ \unity(i\omega_n +\mu) -H^{DFT}(k)   \right]^{-1}_{mm''},
%\end{align}
% and calculate the total number of electrons
%\begin{align}
%  N_e
% &= \lim_{\tau\rightarrow 0^-} \frac{1}{\beta N_k} 
%                \sum_{i\omega_n}\sum_{k,m}G_{0,mm}(k,i\omega_n) \mathrm{e}^{-i\omega_n\tau}.
%\end{align}

\item Make a first guess for the local impurity Selfenergy of the correlated
orbitals $\Sigma^{imp}_{ff}$, for example one can use the GW result
\begin{align}
 \Sigma^{imp}_{ff}(i\omega_n)
 &=  \Sigma^{GW,loc}_{f}(i\omega_n).
\end{align}

\item Set up the interacting {\GF}, where the local component of the GW
Selfenergy for the correlated orbitals $f$ is replaced by $\Sigma^{imp}$
and the Hartree part $\Sigma^{H,imp}_f$ is substracted from $\Sigma^{imp}$. 
In the first step when using $\Sigma^{imp}_{ff} =  \Sigma^{GW,loc}_{f}$ this 
term is zero.
\begin{align}
 G_{mm'}(k,i\omega_n) 
 &= \left[ \unity(i\omega_n+\mu ) -H^{DFT}(k) + v^{XC}(k) \right.\\
          &- \Sigma^{GW}(k,i\omega_n) 
          + \Sigma^{GW,loc}_f(i\omega_n)
          \left. - \Sigma^{imp}(i\omega_n)
          + \Sigma^{H,imp}_f \right]^{-1}_{mm'}.
\end{align}
Adjust the chemical potential $\mu$ in a way such that the desired
filling
\begin{align}
  N_e
 &= \lim_{\tau\rightarrow 0^-} \frac{1}{\beta N_k} 
                \sum_{i\omega_n}\sum_{k,m}G_{mm}(k,i\omega_n) \mathrm{e}^{-i\omega_n\tau}.
\end{align}
is obtained.

\item Calculate the local {\GF} (for all orbitals) then and Weiss field $\mathscr{G}$ ONLY
on the subset of correlated orbitals $f$. Neglecting
offdiagonal components in the hybridization we are also only interested 
in the diagonal components of $\mathscr{G}$
% using the local GW Selfenergy where the diagonal components of the
% correlated orbitals $f$ are replaced by the impurity Selfenergy
\begin{align}
 G^{loc}_{mm'}(i\omega_n) 
 &= \frac{1}{N_k}\sum_k  G_{mm'}(k,i\omega_n) ,\\
%
\Rightarrow \mathscr{G}^{-1}_{ff}(i\omega_n) &= [G^{loc} ]^{-1}_{ff} (i\omega_n)
%                 + \Sigma^{GW,loc}_{ff'}
%                 - \Sigma^{GW,loc}_{ff}
                + \Sigma^{imp}_{ff}(i\omega_n) .
\end{align}
Here, the full $\Sigma^{imp}_{ff}$ has to be used, since 
the Hartree part is calculated in the impurity model.

The Weiss field matrix $\mathscr{G}$ is not explicitly needed, so
we do not invert the last equation to obtain $\mathscr{G}$ .

\item Calculate the Hybridization function
\begin{align}
 \Delta_{ff}(i\omega_n)
 &= i\omega_n + \tilde{\mu}_f -  \mathscr{G}^{-1}_{f	f}(i\omega_n),
\end{align}
where the local chemical potential $\tilde{\mu}_f$ is given by
\begin{align}
 \tilde{\mu}_f = \lim_{\omega_n\rightarrow \infty} \mathrm{Re}\left[ \mathscr{G}^{-1}_{ff}(i\omega_n) \right]
\end{align}
and transform $\Delta(i\omega_n)$ to the imaginary time axis $\tau$ by a Fourier transform
\begin{align}
 \Delta_{ff}(\tau) &= \frac{1}{\beta} \sum_{i\omega_n} \Delta_{ff}(i\omega_n) \mathrm{e}^{-i\omega_n\tau}
\end{align}
and solve the impurity model for the correlated $f$ orbitals.

\item Obtain the new local Selfenergy $\Sigma^{imp}_{ff}(i\omega_n)$ 
from the impurity model and calculate the updated 
Hartree correction from the impurity occupations as given by Eq. \eqref{eq:hartree_selfenergy}
and \eqref{eq:hartree_selfenergy_freq}.
%full {\GF}
%\begin{align}
% G_{mm'}(k,i\omega_n) 
% &= \left[ \unity(i\omega_n+\mu ) -H^{DFT}(k) + v^{XC}_{mm'}(k) \right.\\
%          &- \Sigma^{GW}(k,i\omega_n) 
%          + \Sigma^{GW,loc}_f(i\omega_n)
%          \left. - \Sigma^{imp}(i\omega_n) \right]^{-1}_{mm'}.
%\end{align}
%\begin{align}
%\Sigma^{H,imp}_{f\sigma}
%&=  U_{ff}(n_{f\sigma}+n_{f\bar{\sigma}})
%+ \sum_{f'\neq f} U'_{ff'}n_{f'\bar{\sigma}}
%+ \sum_{f'\neq f} (U'_{ff'}-J_{ff'}) n_{f'\sigma}.
%\end{align}
%Adjust the chemical potential $\mu$ in such a way that the number of electrons
%$N_e$ is preserved.
Then go back to step 3. Repeat until convergence
is reached.

\end{enumerate}


\subsection{Output}
After convergence, {\it e.g.} the local spectral function $A_m(\omega)$
can be obtained by analytic continuation of $ G^{loc}_{mm}(i\omega_n) $.


\section{Hartree- and Exchange term in DMFT}
The derivation follows the ideas of Haule PRL 115, 196403 (2015).

\subsection{Hartree term}
The Hartree energy has the general form
\begin{align}
E^H[\rho]
%
&= \frac{1}{2}\int \mathrm{d}\vec{r}\mathrm{d}\vec{r}' \, 
\rho(\vec{r}) V_C(\vec{r}-\vec{r}') \rho(\vec{r}') \\
%
&= \frac{1}{2} \int \mathrm{d}\vec{r}\mathrm{d}\vec{r}' \,
\frac{\rho(\vec{r}) \rho(\vec{r}') }{|\vec{r}-\vec{r}'|},
\end{align}
where $\rho(\vec{r})$ is the sum of all spin-components
\begin{align}
\rho(\vec{r}) &= \rho_{\uparrow}(\vec{r}) + \rho_{\downarrow}(\vec{r}).
\end{align}

In order to evaluate these term for DMFT we introduce a local orbital basis  $\ket{\chi^{\sigma}_m}$, 
and replace the bare Coulomb interaction $V_C(\vec{r}-\vec{r}')$ 
by an effective screened Coulomb interaction $V_{DMFT}(\vec{r}-\vec{r}')$.
This leads to 
\begin{align}
E^H[\rho]
%
&= \frac{1}{2}\int \mathrm{d}\vec{r}\mathrm{d}\vec{r}' \, 
\rho(\vec{r}) V_C(\vec{r}-\vec{r}') \rho(\vec{r}') \\
%
&= \frac{1}{2}\sum_{\substack{klmn\\\sigma\sigma'}}\int \mathrm{d}\vec{r}\mathrm{d}\vec{r}' 
\braket{\vec{r}|\chi^{\sigma}_k}  \braket{\chi^{\sigma}_k|\rho|\chi^{\sigma}_l}  
\braket{\chi^{\sigma}_l | \vec{r}}V_{DMFT}(\vec{r}-\vec{r}') \nonumber \\
& \hspace{2.7cm} \times 
\braket{\vec{r}'|\chi^{\sigma'}_m}  \braket{\chi^{\sigma'}_m|\rho|\chi^{\sigma'}_n}  
\braket{\chi^{\sigma'}_n | \vec{r}'} \\
%
&= \frac{1}{2}\sum_{\substack{klmn\\\sigma\sigma'}}\int \mathrm{d}\vec{r}\mathrm{d}\vec{r}' 
 (\chi^{\sigma}_l)^*(\vec{r}) (\chi^{\sigma'}_n)^*(\vec{r}')
    V_{DMFT}(\vec{r}-\vec{r}') 
  \chi^{\sigma'}_m(\vec{r}')  \chi^{\sigma}_k(\vec{r}) \nonumber \\
& \hspace{2.7cm} \times 
\braket{\chi^{\sigma}_k|\rho|\chi^{\sigma}_l} 
\braket{\chi^{\sigma'}_m|\rho|\chi^{\sigma'}_n}  .
\end{align}
%
In the last equation we can now identify the matrix elements of the local screened Coulomb
interaction
\begin{align}
\braket{ln|U|km}
&=
\int \mathrm{d}\vec{r}\mathrm{d}\vec{r}' 
 (\chi^{\sigma}_l)^*(\vec{r}) (\chi^{\sigma'}_n)^*(\vec{r}')
    V_{DMFT}(\vec{r}-\vec{r}') 
  \chi^{\sigma'}_m(\vec{r}')  \chi^{\sigma}_k(\vec{r}) ,
\end{align}
and the DMFT density matrix
\begin{align}
\braket{\chi^{\sigma}_k|\rho|\chi^{\sigma}_l} 
&= n^{\sigma}_{kl}.
\end{align}
With this, the Hartree energy takes on the form
\begin{align}
E^{DMFT}
&= \frac{1}{2}\sum_{\substack{klmn\\\sigma\sigma'}}
\braket{ln|U|km}
n^{\sigma}_{kl} n^{\sigma'}_{mn}.
\end{align}
In the impurity model we restrict ourselves to diagonal density matrices, which leads to
\begin{align}
E^H_{DMFT}
&= \frac{1}{2}\sum_{\substack{km\\\sigma\sigma'}}
\braket{km|U|km}
n^{\sigma}_{k} n^{\sigma'}_{m}.
\end{align}
This leads to the following Hartree part of the Selfenergy in DMFT
\begin{align}
\Sigma^{H,DMFT}_{l\sigma}
&= \frac{\partial }{\partial n^{\sigma}_l} E^H_{DMFT} \\
&= \frac{1}{2} \sum_{m,\sigma'} \braket{lm|U|lm} n^{\sigma'}_{m}
+
\frac{1}{2} \sum_{k,\sigma'} \braket{kl|U|kl} n^{\sigma'}_{k} \\
%
&= \sum_{m,\sigma'} \braket{lm|U|lm} n^{\sigma'}_{m} \\
%
&= U_0 (n^{\uparrow}_l + n^{\downarrow}_l)
               + \sum_{m\neq l} (U_0-2J_{lm}) (n^{\uparrow}_m + n^{\downarrow}_m) .
\label{eq:hartree_selfenergy}
\end{align}


\subsection{Exchange term}
The exact exchange energy has the general form
\begin{align}
E^X[\rho]
%
&= -\frac{1}{2}\sum_{\sigma} \int \mathrm{d}\vec{r}\mathrm{d}\vec{r}' \, 
\rho_{\sigma}(\vec{r},\vec{r}') V_C(\vec{r}-\vec{r}') \rho_{\sigma}(\vec{r}',\vec{r}) \\
%
&= -\frac{1}{2}\sum_{\sigma} \int \mathrm{d}\vec{r}\mathrm{d}\vec{r}' \,
\frac{\rho_{\sigma}(\vec{r},\vec{r}') \rho_{\sigma}(\vec{r}',\vec{r}) }{|\vec{r}-\vec{r}'|}.
\end{align}
In order to evaluate these term for DMFT we introduce a local orbital basis  $\ket{\chi^{\sigma}_m}$, 
and replace the bare Coulomb interaction $V_C(\vec{r}-\vec{r}')$ 
by an effective screened Coulomb interaction $V_{DMFT}(\vec{r}-\vec{r}')$.
This leads to 
\begin{align}
E^X[\rho]
&= -\frac{1}{2}\sum_{\sigma} \int \mathrm{d}\vec{r}\mathrm{d}\vec{r}' \, 
\rho_{\sigma}(\vec{r},\vec{r}') V_{DMFT}(\vec{r}-\vec{r}') \rho_{\sigma}(\vec{r}',\vec{r}) \\
%
&= -\frac{1}{2} \sum_{\substack{klmn\\\sigma}} \int \mathrm{d}\vec{r}\mathrm{d}\vec{r}' \, 
\braket{\vec{r} | \chi^{\sigma}_k }\braket{\chi^{\sigma}_k | \rho_{\sigma} | \chi^{\sigma}_l } \braket{\chi^{\sigma}_l | \vec{r}'}  V_{DMFT}(\vec{r}-\vec{r}') \nonumber\\
& \hspace{3.8cm} \times 
\braket{\vec{r}' | \chi^{\sigma}_m }\braket{\chi^{\sigma}_m | \rho_{\sigma} | \chi^{\sigma}_n } \braket{\chi^{\sigma}_n | \vec{r}} \\
% 
&= -\frac{1}{2} \sum_{\substack{klmn\\\sigma}} \int \mathrm{d}\vec{r}\mathrm{d}\vec{r}' \, 
(\chi^{\sigma}_n)^*(\vec{r}) (\chi^{\sigma}_l)^*(\vec{r}')  
V_{DMFT}(\vec{r}-\vec{r}') 
\chi^{\sigma}_m (\vec{r}') \chi^{\sigma}_k (\vec{r}) \nonumber\\
& \hspace{3.8cm} \times 
\braket{\chi^{\sigma}_m | \rho_{\sigma} | \chi^{\sigma}_n }
\braket{\chi^{\sigma}_k | \rho_{\sigma} | \chi^{\sigma}_l }.
\end{align}
In the last equation we can now identify the matrix elements of the local screened Coulomb
interaction
\begin{align}
\braket{nl|U|km} 
&= \int \mathrm{d}\vec{r}\mathrm{d}\vec{r}' \, 
(\chi^{\sigma}_n)^*(\vec{r}) (\chi^{\sigma}_l)^*(\vec{r}')  
V_{DMFT}(\vec{r}-\vec{r}') 
\chi^{\sigma}_m (\vec{r}') \chi^{\sigma}_k (\vec{r}),
\end{align}
and the DMFT density matrix
\begin{align}
\braket{\chi^{\sigma}_m | \rho_{\sigma} | \chi^{\sigma}_n } 
&= n^{\sigma}_{mn}.
\end{align}
With this, the exchange energy takes on the form
\begin{align}
E^X_{DMFT}
&= -\frac{1}{2} \sum_{\substack{klmn\\\sigma}}\braket{nl|U|km} n^{\sigma}_{mn} n^{\sigma}_{kl}.
\end{align}
In the impurity model we restrict ourselves to diagonal density matrices, which leads to
\begin{align}
E^X_{DMFT}
&= -\frac{1}{2} \sum_{mk,\sigma}\braket{mk|U|km} n^{\sigma}_{m} n^{\sigma}_{k} .
\end{align}
This leads to the following exchange part of the Selfenergy in DMFT
\begin{align}
\Sigma^{X,DMFT}_{l\sigma}
&= \frac{\partial }{\partial n^{\sigma}_l} E^X_{DMFT} \\
%
&= -\frac{1}{2} \sum_{k}\braket{lk|U|kl} n^{\sigma}_{k} 
   -\frac{1}{2} \sum_{m}\braket{ml|U|lm} n^{\sigma}_{m}  \\
%
&= - \sum_{k}\braket{lk|U|kl} n^{\sigma}_{k} \\
%
&= - U_0\, n^{\sigma}_l -  \sum_{k\neq l} J_{lk} \, n^{\sigma}_{k}.
\end{align}

\subsection{Hartree + exchange Selfenergy}
For consistency checks, we add up the Hartree and the exchange contribution
to the Selfenergy and obtain
\begin{align}
\Sigma^{H,DMFT}_{l\sigma} + \Sigma^{X,DMFT}_{l\sigma}
%
&= U_0 n^{\bar{\sigma}}_l 
%
               + \sum_{m\neq l} (U_0-2J_{lm}) n^{\bar{\sigma}}_m  \nonumber \\
& \hspace{1.4cm} + \sum_{m\neq l} (U_0-3J_{lm}) n^{     \sigma }_m  \\
&= \lim_{\omega_n\rightarrow \infty} \Sigma^{DMFT}(i\omega_n),
\end{align}
which is identical to the high-frequency limit of the true DMFT Selfenergy.
This term is also equal to the sum of all first order diagrams to the
DMFT Selfenergy, i.e. the Hartree- and the Fock diagram.

\subsection{Dynamical interactions}
In the case of dynamical interactions $U(\omega)$ in the Hartree and exchange part
the screened Coulomb matrix elements recover their bare values \cng{is this correct?}, i.e.
$U_0$ has to be replaced by the bare $V$ (assuming no frequency dependence of the 
Hund's coupling)
\begin{align}
\Sigma^{H,DMFT}_{l\sigma}
&= V (n^{\uparrow}_l + n^{\downarrow}_l)
               + \sum_{m\neq l} (V-2J_{lm}) (n^{\uparrow}_m + n^{\downarrow}_m) 
\label{eq:hartree_selfenergy_freq} \\
%
\Sigma^{X,DMFT}_{l\sigma}
&= - V\, n^{\sigma}_l -  \sum_{k\neq l} J_{lk} \, n^{\sigma}_{k}.               
\end{align}
\cng{CAUTION!} Does $U_0$ or $F_0$ recover the bare interaction? If $F_0=V$,
then $U_0$ has to be replaced in the 5-orbital model by $V + \frac{8}{7}J_{avg}$.


\section{Implementation details}

\subsection{Impurity solver input}

The CT-HYB impurity solver by Yusuke needs the following input files
\begin{description}
\item[dmft.input] Includes information about U,J, number of frequencies, etc. At the moment
possible: Only 3-fold degenerate orbitals. No freq. dependent U.

\item[hyb\_tau.dat] The hybridization function as a matrix for imaginary time. 
It needs to be diagonal! \\
Only real part, one column. Seperate matrix elements
via two line breaks and \verb|# hyb     2    1| etc. We need \verb|Nmesh+1|
points where the endpoints $\tau=0,\beta$ are included! By convention has negative sign.
\cng{The local orbital levels are assumed to be $\tilde{\mu}=0$ and any
shift is absorbed in the chemical potential! This has to 
be checked for consistency!!!}

\item[omega\_mesh.dat] Specifies the bosonic frequency grid for some 
correlation functions. Just reuse the standard template file. Not important for us.

\item[fort.10*] Includes information about the Monte-Carlo configuration used
for starting the sampling. Is initialized once with Yusuke's code and then overwritten by 
the solver. No change required here.
\end{description}


\end{document}
