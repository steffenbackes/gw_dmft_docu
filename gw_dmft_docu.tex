\documentclass[12pt,a4paper]{scrartcl}
\usepackage[utf8]{inputenc}
\usepackage{amsmath}
\usepackage{amssymb}
\usepackage{braket}
\usepackage{amssymb}
\usepackage{graphicx}
\usepackage{float}
\usepackage{wrapfig}
\usepackage{dsfont}
\usepackage{enumitem}
\usepackage{subfig}
\usepackage[top=3cm, left=3cm, right=3cm, bottom=3cm]{geometry}
\usepackage{fancyhdr}
\usepackage{sidecap}
\usepackage{pstricks}
\usepackage{mathrsfs}
\usepackage{listings}
\usepackage[absolute]{textpos}
\numberwithin{equation}{subsection}
%\numberwithin{figure}{section}
%\usepackage[pdfborder=0 0 0]{hyperref}
\usepackage[colorlinks=True, urlcolor=blue]{hyperref}
\pagestyle{fancy}
\fancyhead[C]{}
\fancyhead[L]{}
\fancyfoot[C]{\thepage}
\fancyhead[R]{}

\newcommand{\cng}[1]{{\color{red}#1}}
\newcommand{\sgn}{{\mathrm{sgn}}}
\newcommand{\GF}{Green's function}
\newcommand{\unity}{\mathds{1}}


\begin{document}

\section{GW (oneshot) + DMFT documentation}

This document  provides the prescription of the combination of a GW calculation
for correlated materials with DMFT applied only to a subset of correlated orbitals.
At this level, the GW calculation will be performed only once at the beginning (one-shot)
based on a DFT Hamiltonian $H^{DFT}$,
to obtain a nonlocal Selfenergy $\Sigma^{GW}$ for all states. 
In addition, the local part of $\Sigma^{GW}$ of a subset of strongly correlated orbitals will be replaced by a Selfenergy $\Sigma^{DMFT}$
obtained within a selfconsistent DMFT scheme, 
where the selfconsistency is done including the full nonlocal
effects of the combined Selfenergy.

No further selfconsistency apart from the DMFT cycle will be performed, {\it i.e.} no update of $\Sigma^{GW}$ will be done. By this, the final interacting system will be
described by a {\GF} with the non-interacting DFT dispersion, corrected by a non-local Selfenergy where the non-local components
correspond to $\Sigma^{GW}=G_0W_0$, while the local components
of the correlated orbitals correspond to $\Sigma^{DMFT}$.
This $\Sigma^{DMFT}$ is usually different to the one obtained by
a standard DMFT calculation since the selfconsistency is
done with the inclusion of the nonlocal parts of the 
Selfenergy.

Extensions to a full GW+DMFT selfconsistency will be discussed elsewhere.


\subsection{The GW part}

On the basis of $H^{DFT}$ a $G_0W_0$ calculation has to be performed on the full
system. By this, the Selfenergy in the Kohn-Sham basis is obtained for all states
\begin{align}
 \Sigma_{\nu\nu'}(k,\omega)
 &= \left[ G_0W_0  \right]_{\nu\nu'}(k,\omega).
\end{align}
By this, the GW estimate for the full interacting {\GF} is given by
\begin{align}
 G^{GW}_{\nu\nu'}(k,\omega) 
 &= \left[ \unity(\omega +\mu +i\delta) -H^{DFT}(k)+v^{XC}(k) - \Sigma^{GW}(k,\omega)  \right]^{-1}_{\nu\nu'}.
\end{align}


\subsubsection{Output}
At this point the basis transformation to the local Wannier basis will be
performed on the GW side.
For the next step of the DMFT calculation one needs
on a mesh in k-space in the full Brillouin zone:
\begin{itemize}
\item $\epsilon_m(k)$: The eigenvalues  of $H^{DFT}(k)$ in the Wannier basis
for all relevant orbitals
\item $\mu$: The chemical potential that yields the correct physical number of electrons $N_e$. It is not needed if all $\epsilon_m(k)$ are given 
with respect to the Fermi level.
for $H^{DFT}$
\item $v^{XC}_{mm'}(k)$: The value of the exchange-correlation potential 
in the Wannier basis for all relevant orbitals
\item $\Sigma^{GW}_{mm'}(k,i\omega)$: The Selfenergy within GW in the Wannier basis for all relevant orbitals on imaginary frequencies $\omega$.
\item $\beta$: The inverse temperature used for defining $\omega_n=(2n+1)\pi/\beta$.
\end{itemize}
{\it All output from the GW calculation will be in atomic units and
have to be converted to eV!!}


\subsection{The DMFT part}

Within DMFT we then calculate a local correction $\Sigma^{DMFT}$ for a subset of correlated
Wannier orbitals.

The input of the calculation will be the output of the GW calculation. First, one will usually apply a Wannier-interpolation of the GW data to obtain a fine k-mesh
since the GW output will be given on a very coarse grid.

Since the Hartree term is already incorporated on the DMFT level, 
it has to be excluded from the Selfenergies in GW (already taken care of)
and in DMFT (has to be done in the selfconsistency).

We then proceed as follows:

\begin{enumerate}

%\item Transform the Selfenergy from GW to the imaginary Matsubara axis by
%\begin{align}
%\Sigma^{GW}_{mm'}(k,i\omega_n)
%&= \frac{1}{2\pi i} \int \, \frac{\Sigma^{GW}_{mm'}(k,\omega)}{\omega - i\omega_n}\ \mathrm{d}\omega , \mbox{\hspace{1cm} or}\\
%&= \frac{1}{\pi} \int \, \frac{\mathrm{Im}[\Sigma^{GW}_{mm'}(k,\omega)]}{\omega - i\omega_n} \ \mathrm{d}\omega , \mbox{\hspace{0.7cm} or} \\
%&= \frac{1}{\pi i} \int \, \frac{\mathrm{Re}[\Sigma^{GW}_{mm'}(k,\omega)]}{\omega - i\omega_n} \ \mathrm{d}\omega .
%\end{align}

\item Calculate the local diagonal part of the GW Selfenergy ONLY for the correlated subset of orbitals $f$ that will be later replaced by the DMFT result
\begin{align}
 \Sigma^{GW,loc}_{f}(i\omega_n)
 &= \frac{1}{N_k}\sum_k \Sigma^{GW}_{ff}(k,i\omega_n).
\end{align}


%\item (This step can be omitted if the correct physical number of electrons $N_e$ is already known)
% 
%Construct the initial non-interacting {\GF} on the imaginary Matsubara axis via
%\begin{align}
%   G_{0,mm'}(k,i\omega_n) 
% &= \left[ \unity(i\omega_n +\mu) -H^{DFT}(k)   \right]^{-1}_{mm''},
%\end{align}
% and calculate the total number of electrons
%\begin{align}
%  N_e
% &= \lim_{\tau\rightarrow 0^-} \frac{1}{\beta N_k} 
%                \sum_{i\omega_n}\sum_{k,m}G_{0,mm}(k,i\omega_n) \mathrm{e}^{-i\omega_n\tau}.
%\end{align}

\item Make a first guess for the local impurity Selfenergy of the correlated
orbitals $\Sigma^{imp}_{ff}$, for example one can use the GW result
\begin{align}
 \Sigma^{imp}_{ff}(i\omega_n)
 &=  \Sigma^{GW,loc}_{f}(i\omega_n).
\end{align}

\item Set up the interacting {\GF}, where the local component of the GW
Selfenergy for the correlated orbitals $f$ is replaced by $\Sigma^{imp}$
and the Hartree part $\Sigma^{H,imp}_f$ is substracted from $\Sigma^{imp}$. 
In the first step when using $\Sigma^{imp}_{ff} =  \Sigma^{GW,loc}_{f}$ this 
term is zero.
\begin{align}
 G_{mm'}(k,i\omega_n) 
 &= \left[ \unity(i\omega_n+\mu ) -H^{DFT}(k) + v^{XC}(k) \right.\\
          &- \Sigma^{GW}(k,i\omega_n) 
          + \Sigma^{GW,loc}_f(i\omega_n)
          \left. - \Sigma^{imp}(i\omega_n)
          + \Sigma^{H,imp}_f \right]^{-1}_{mm'}.
\end{align}
Adjust the chemical potential $\mu$ in a way such that the desired
filling
\begin{align}
  N_e
 &= \lim_{\tau\rightarrow 0^-} \frac{1}{\beta N_k} 
                \sum_{i\omega_n}\sum_{k,m}G_{mm}(k,i\omega_n) \mathrm{e}^{-i\omega_n\tau}.
\end{align}
is obtained.

\item Calculate the local {\GF} (for all orbitals) then and Weiss field $\mathscr{G}$ ONLY
on the subset of correlated orbitals $f$. Neglecting
offdiagonal components in the hybridization we are also only interested 
in the diagonal components of $\mathscr{G}$
% using the local GW Selfenergy where the diagonal components of the
% correlated orbitals $f$ are replaced by the impurity Selfenergy
\begin{align}
 G^{loc}_{mm'}(i\omega_n) 
 &= \frac{1}{N_k}\sum_k  G_{mm'}(k,i\omega_n) ,\\
%
\Rightarrow \mathscr{G}^{-1}_{ff}(i\omega_n) &= [G^{loc} ]^{-1}_{ff} (i\omega_n)
%                 + \Sigma^{GW,loc}_{ff'}
%                 - \Sigma^{GW,loc}_{ff}
                + \Sigma^{imp}_{ff}(i\omega_n) .
\end{align}
Here, the full $\Sigma^{imp}_{ff}$ has to be used, since 
the Hartree part is calculated in the impurity model.

The Weiss field matrix $\mathscr{G}$ is not explicitly needed, so
we do not invert the last equation to obtain $\mathscr{G}$ .

\item Calculate the Hybridization function
\begin{align}
 \Delta_{ff}(i\omega_n)
 &= i\omega_n + \tilde{\mu}_f -  \mathscr{G}^{-1}_{ff}(i\omega_n),
\end{align}
where the local chemical potential $\tilde{\mu}_f$ is given by
\begin{align}
 \tilde{\mu}_f = \lim_{\omega_n\rightarrow \infty} \mathrm{Re}\left[ \mathscr{G}^{-1}_{ff}(i\omega_n) \right]
\end{align}
and transform $\Delta(i\omega_n)$ to the imaginary time axis $\tau$ by a Fourier transform
\begin{align}
 \Delta_{ff}(\tau) &= \frac{1}{\beta} \sum_{i\omega_n} \Delta_{ff}(i\omega_n) \mathrm{e}^{-i\omega_n\tau}
\end{align}
and solve the impurity model for the correlated $f$ orbitals.

\item Obtain the new local Selfenergy $\Sigma^{imp}_{ff}(i\omega_n)$ 
from the impurity model and calculate the updated 
Hartree correction
%full {\GF}
%\begin{align}
% G_{mm'}(k,i\omega_n) 
% &= \left[ \unity(i\omega_n+\mu ) -H^{DFT}(k) + v^{XC}_{mm'}(k) \right.\\
%          &- \Sigma^{GW}(k,i\omega_n) 
%          + \Sigma^{GW,loc}_f(i\omega_n)
%          \left. - \Sigma^{imp}(i\omega_n) \right]^{-1}_{mm'}.
%\end{align}
\begin{align}
\Sigma^{H,imp}_{f\sigma}
&=  U_{ff}(n_{f\sigma}+n_{f\bar{\sigma}})
+ \sum_{f'\neq f} U'_{ff'}n_{f'\bar{\sigma}}
+ \sum_{f'\neq f} (U'_{ff'}-J_{ff'}) n_{f'\sigma}.
\end{align}
%Adjust the chemical potential $\mu$ in such a way that the number of electrons
%$N_e$ is preserved.
Then go back to step 3. Repeat until convergence
is reached.

\end{enumerate}


\subsubsection{Output}
After convergence, {\it e.g.} the local spectral function $A_m(\omega)$
can be obtained by analytic continuation of $ G^{loc}_{mm}(i\omega_n) $.


\end{document}
